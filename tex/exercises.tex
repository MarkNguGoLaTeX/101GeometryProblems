\part{101 bài toán}

\setcounter{section}{0}

    \section{Đề bài}

    \newpage

    \section{Lời giải}

        \begin{problem}
            Cho góc \(Oxy\) và điểm \(P\) nằm trong góc cố định. Đường tròn thay đổi luôn đi qua hai điểm \(O\), \(P\) cắt \(Ox\), \(Oy\) tại \(M\), \(N\). Tìm quỹ tích trọng tâm \(G\) và trực tâm \(H\) của tam giác \(PMN\).
        \end{problem}

        \begin{center}
            \begin{tikzpicture}[line cap=round,line join=round,>=triangle 45,x=1cm,y=1cm,scale=0.85]
                \draw [line width=0.4pt] (0,0)-- (7.146911818448885,1.7867462802928267);
                \draw [line width=0.4pt] (2.9835094959629713,3.2531345901598625) circle (4.41409261051417cm);
                \draw [line width=0.4pt] (7.146911818448885,1.7867462802928267)-- (5.967018991925942,0);
                \draw [line width=0.4pt] (5.967018991925942,0)-- (6.866238328572519,5.352808466412626);
                \draw [line width=0.4pt] (7.146911818448885,1.7867462802928267)-- (6.866238328572519,5.352808466412626);
                \draw [line width=0.4pt] (7.146911818448885,1.7867462802928267)-- (6.416628660249231,2.676404233206313);
                \draw [line width=0.4pt] (6.866238328572519,5.352808466412626)-- (6.556965405187413,0.8933731401464133);
                \draw [line width=0.4pt] (5.967018991925942,0)-- (7.006575073510701,3.5697773733527267);
                \draw [line width=0.4pt] (5.778099495723505,1.4077037474272214) circle (1.420324053364592cm);
                \draw [line width=0.4pt] (8.561067895704717,3.69212659291799) circle (2.372827753246992cm);
                \draw [line width=0.4pt] (0,0)-- (12.458574206199849,4.140886179491225);
                \draw [line width=0.4pt] (7.146911818448885,0)-- (5.311662387750965,4.1408861794912255);
                \draw [line width=0.4pt] (7.146911818448885,1.7867462802928267)-- (5.311662387750965,4.1408861794912255);
                \draw [line width=0.4pt] (7.146911818448885,1.7867462802928267)-- (7.146911818448885,0);
                \draw [line width=0.4pt] (7.146911818448885,1.7867462802928267)-- (14.013150147021397,0.633285566385727);
                \draw [line width=0.4pt] (14.013150147021397,0.633285566385727)-- (6.866238328572519,5.352808466412626);
                \draw [line width=0.4pt] (14.013150147021397,0.633285566385727)-- (5.967018991925942,0);
                \draw [line width=0.4pt] (7.146911818448885,1.7867462802928267)-- (12.458574206199849,4.140886179491225);
                \draw [line width=0.4pt] (12.458574206199849,4.140886179491225)-- (7.146911818448885,0);
                \draw [line width=0.4pt] (12.458574206199849,4.140886179491225)-- (5.311662387750965,4.1408861794912255);
                \draw [line width=0.4pt] (7.903907014121772,7.487044785524923)-- (5.199991313876848,-1.2588409350467702);
                \draw [line width=0.4pt] (10.311125031296275,8.986190834421118)-- (14.84973592927374,-1.2543088094753236);
                \draw [line width=0.4pt] (0,0)-- (11.392789342276075,8.881634503367039);
                \draw [line width=0.4pt] (0,0)-- (15.53787883812214,0);
                \draw [line width=0.4pt] (3.5734559092244425,0.8933731401464133) -- (2.9835094959629713,3.2531345901598625);
                \draw [line width=0.4pt] (7.146911818448885,1.7867462802928267) -- (2.9835094959629713,3.2531345901598625);
                \begin{scriptsize}
                    \draw [fill=black] (0,0) circle (0.6pt);
                    \draw[color=black] (0.15697208380035932-0.3,0.2971479021457066-0.6) node {$O$};
                    \draw [fill=black] (7.146911818448885,1.7867462802928267) circle (0.6pt);
                    \draw[color=black] (7.295848187647818+0.1,2.0865759492842324) node {$P$};
                    \draw [fill=black] (2.9835094959629713,3.2531345901598625) circle (0.6pt);
                    \draw[color=black] (3.1330734674623604-0.2,3.5369544717017742) node {$I$};
                    \draw [fill=black] (5.967018991925942,0) circle (0.6pt);
                    \draw[color=black] (6.1091748511243615,0.2971479021457066-0.6) node {$M$};
                    \draw [fill=black] (6.866238328572519,5.352808466412626) circle (0.6pt);
                    \draw[color=black] (7.013306917046995,5.6465959588545624) node {$N$};
                    \draw [fill=black] (6.416628660249231,2.676404233206313) circle (0.6pt);
                    \draw[color=black] (6.561240884085678,2.9718719305001344) node {$A$};
                    \draw [fill=black] (6.556965405187413,0.8933731401464133) circle (0.6pt);
                    \draw [fill=black] (7.006575073510701,3.5697773733527267) circle (0.6pt);
                    \draw [fill=black] (6.660056379649116,2.3798515822351525) circle (0.6pt);
                    \draw[color=black] (6.806109985273058+0.1,2.6704945751925933) node {$G$};
                    \draw [fill=black] (14.013150147021397,0.633285566385727) circle (0.6pt);
                    \draw[color=black] (14.171019105601175+0.1,0.9187386974675102-0.1) node {$H$};
                    \draw [fill=black] (7.146911818448885,0) circle (0.6pt);
                    \draw[color=black] (7.295848187647818-0.2,0.2971479021457066-0.6) node {$B$};
                    \draw [fill=black] (5.311662387750965,4.1408861794912255) circle (0.6pt);
                    \draw[color=black] (5.468747971095829-0.2,4.441086537624398) node {$C$};
                    \draw [fill=black] (5.589179999521081,0) circle (0.6pt);
                    \draw[color=black] (5.732453156989931-0.4,0.2971479021457066-0.6) node {$K$};
                    \draw [fill=black] (7.364062702018223,5.740904304839593) circle (0.6pt);
                    \draw[color=black] (7.521881204128476+0.1,6.04215373769571+0.1) node {$L$};
                    \draw [fill=black] (12.458574206199849,4.140886179491225) circle (0.6pt);
                    \draw[color=black] (12.607624074943288+0.1,4.441086537624398-0.1) node {$J$};
                    \draw [fill=black] (3.5734559092244425,0.8933731401464133) circle (0.6pt);
                    \draw[color=black] (3.716992093370728,1.1824438833616087-0.6) node {$D$};
                    \draw[color=black] (11.552803331366883,9.168943799011451) node {$y$};
                    \draw[color=black] (15.69674196684562,0.2971479021457066) node {$x$};
                \end{scriptsize}
            \end{tikzpicture}
        \end{center}

        \begin{solution}
            Ta chia bài toán ra thành hai phần.
            
            \begin{enumerate}
            
                \item[(a)] \textit{Tìm quỹ tích trọng tâm \(G\) của tam giác \(PMN\).}

                Gọi \(A\) là trung điểm của đoạn thẳng \(MN\). Do \(G\) là trọng tâm của tam giác \(PMN\) nên \(\overrightarrow{PA} = \dfrac{3}{2}\overrightarrow{PG}\). Ta sẽ đi tìm quỹ tích của điểm \(A\) khi đường tròn đi qua hai điểm \(O\) và \(P\) thay đổi.

                \begin{enumerate}[leftmargin=1.25cm]
                
                    \item[Thuận.] Gọi \(K\) là giao điểm của \((APM)\) và \(Ox\), \(L\) là giao điểm của \((APN)\) và \(Oy\).\\
                    Ta có \(\angle PAL = \angle PNL = \angle PMK = 180 \degree - \angle PAK\). Do đó \(K\), \(A\), \(L\) thẳng hàng. Mặt khác, \(\angle PKL = \angle PKA = \angle PMA = \angle PMN = \angle POy\) và \(\angle PLK = \angle PLA = \angle PNA = \angle PNM = \angle POx\). Mà \(\{K\} \in Ox\) và \(\{L\} \in Oy\) nên \(K\) và \(L\) xác định duy nhất, hay \(K\) và \(L\) là hai điểm cố định. Do đó \(A\) thuộc đường thẳng cố định; đường thẳng đi qua \(K\) và \(L\).

                    \item[Đảo.] Chọn \(K\), \(L\) sao cho \(\{K\} \in Ox\), \(\{L\} \in Oy\), \(\angle PKL = \angle POy\), \(\angle PLK = \angle POx\). Lấy điểm \(A\) bất kì trên \(KL\). Gọi \(M\) là giao điểm của \((APK)\) và \(Ox\), \(N\) là giao điểm của \((APL)\) và \(Oy\).\\
                    Ta có \(\angle PAM = \angle PKM = \angle PLO = 180 \degree - \angle PAN\). Do đó \(M\), \(A\), \(N\) thẳng hàng. Mặt khác, \(\angle PMN = \angle PMA = \angle PKA = \angle PKL = \angle POy\) và \(\angle PNM = \angle PNA = \angle PLa = \angle PLK = \angle POx\). Suy ra \(M\), \(N\), \(O\), \(P\) đồng viên. Do đó, tồn tại hai điểm \(M\) và \(N\) sao cho \(\{M\} \in Ox\), \(\{N\} \in Oy\), \(\{A\} \in MN\) và \(M\), \(N\) thuộc đường tròn đi qua hai điểm \(O\) và \(P\).
                
                \end{enumerate}

                Vì vậy, quỹ tích của điểm \(A\) là một đường thẳng cố định, đi qua hai điểm \(K\) và \(L\) được xác định như trên. Do điểm \(G\) là ảnh của \(A\) qua phép vị tự tâm \(P\), quỹ tích của điểm \(G\) là một đường thẳng song song với đường thẳng cố định nói trên.

                \item[(b)] \textit{Tìm quỹ tích trực tâm \(H\) của tam giác \(PMN\).}

                Gọi \(B\), \(C\) lần lượt là hình chiết của \(P\) lên \(Ox\), \(Oy\); \(D\) là trung điểm của đoạn thẳng \(OP\); \(J\) là trực tâm tam giác \(PBC\).\\
                Ta có \(CJ \perp PB\) và \(PB \perp OB\), do đó \(CJ \parallel OB\). Tương tự, \(BJ \parallel OC\). Suy ra tứ giác \(OBJC\) là hình bình hành, hay \(J\) là điểm cố định.

                \begin{enumerate}[leftmargin=1.25cm]
                
                    \item[Thuận.] Gọi \(H\) và \(I\) lần lượt là trực tâm và tâm đường tròn ngoại tiếp tam giác \(PMN\).\\
                    Biến đổi góc, ta được \(\triangle PBC \sim \triangle PMN\). Ta cũng có \(D\), \(I\) lần lượt là tâm đường tròn ngoại tiếp các tam giác \(PBC\), \(PMN\), và \(J\), \(H\) lần lượt là trực tâm các tam giác \(PBC\), \(PMN\). Do đó, xét phép vị tự quay
                    \[\mathcal{F}_{P}: B \mapsto M, C \mapsto N, D \mapsto I, J \mapsto H\]
                    Hay \(\triangle PDI \sim \triangle PJH\). Mà \(\angle IDP = 90 \degree\) nên \(\angle PJH = 90 \degree\). Nói cách khác, \(H\) thuộc đường thẳng vuông góc \(PJ\) tại \(J\); đây là một đường thẳng cố định.

                    \item[Đảo.] Gọi \(H\) là điểm bất kì thuộc đường thẳng vuông góc \(PJ\) tại \(J\). Chọn \(I\) thuộc đường trung trực của đoạn thẳng \(OP\) sao cho \(\angle IPD = \angle JPH\). Đường tròn \(I;IO\) cắt \(Ox\), \(Oy\) lần lượt tại \(M\), \(N\).\\
                    Biến đổi góc, ta được \(\triangle PBC \sim \triangle PMN\). Hơn nữa, các tam giác \(PDI\) và \(JPH\) vuông và đồng dạng. Do đó, xét phép vị tự quay
                    \[\mathcal{F}_{P}: B \mapsto M, C \mapsto N, D \mapsto I, J \mapsto H\]
                    Mà \(J\) là trực tâm tam giác \(PBC\) nên \(H\) là trực tâm tam giác \(PMN\). Do đó, tồn tại hai điểm \(M\) và \(N\) sao cho \(\{M\} \in Ox\), \(\{N\} \in Oy\), \(H\) là trực tâm tam giác \(PMN\) và \(M\), \(N\) thuộc đường tròn đi qua hai điểm \(O\) và \(P\).
                
                \end{enumerate}

                Vì vậy, quỹ tích của điểm \(H\) là một đường thẳng cố định vuông góc với đường thẳng \(PJ\) tại \(J\), với điểm \(J\) được xác định như trên.
                
            \end{enumerate}

        Bài toán quỹ tích được giải quyết.
        \end{solution}

        \begin{problem}
            Cho đường tròn \((O)\) và một dây cung \(AB\) cố định không là đường kính. Một điểm \(P\) thay đổi trên cung lớn \(AB\). Gọi \(I\) là trung điểm của \(AB\). Lấy các điểm \(M\), \(N\) trên các tia \(PA\), \(PB\) tương ứng sao cho \(\angle PMI = \angle PNI = \angle APB\).
            \begin{enumerate}
                \item[(a)] Chứng minh rằng đường cao kẻ từ \(P\) của tam giác \(PMN\) luôn đi qua một điểm cố định.
                \item[(b)] Chứng minh rằng đường thẳng Euler của tam giác \(PMN\) luôn đi qua một điểm cố định.
            \end{enumerate}
        \end{problem}

        \begin{center}
            \begin{tikzpicture}[line cap=round,line join=round,>=triangle 45,x=1cm,y=1cm,scale=1.2]
                \draw [line width=0.4pt] (0,0) circle (3cm);
                \draw [line width=0.4pt] (-2.2582018253534843,-1.974974560841276)-- (2.2821549054241452,-1.9472465143500735);
                \draw [line width=0.4pt] (0.01197654003533044,-1.9611105375956748)-- (-2.599229470653089,-3.179605347001402);
                \draw [line width=0.4pt] (0.01197654003533044,-1.9611105375956748)-- (2.50614491728459,-2.2873197464715567);
                \draw [line width=0.4pt] (-2.599229470653089,-3.179605347001402)-- (2.50614491728459,-2.2873197464715567);
                \draw [line width=0.4pt] (-0.8878452142101361,2.8656117803366454)-- (-2.2582018253534843,-1.974974560841276);
                \draw [line width=0.4pt] (-0.8878452142101361,2.8656117803366454)-- (2.2821549054241452,-1.9472465143500735);
                \draw [line width=0.4pt] (0.028025510772083744,-4.589065317236954)-- (-2.2582018253534843,-1.974974560841276);
                \draw [line width=0.4pt] (0.028025510772083744,-4.589065317236954)-- (2.2821549054241452,-1.9472465143500735);
                \draw [line width=0.4pt] (0,0)-- (0.028025510772083744,-4.589065317236954);
                \draw [line width=0.4pt] (0.01197654003533044,-1.9611105375956748)-- (1.7554294096888614,-1.1475445977883159);
                \draw [line width=0.4pt] (0.01197654003533044,-1.9611105375956748)-- (-2.1733627213725577,-1.675292708257512);
                \draw [line width=0.4pt] (1.7554294096888614,-1.1475445977883159)-- (-2.1733627213725577,-1.675292708257512);
                \draw [line width=0.4pt] (0.01898291772835164,-3.108376866904913) circle (2.6191810947414282cm);
                \draw [line width=0.4pt] (-2.1733627213725577,-1.675292708257512)-- (0.8091498515372272,0.2891460169325444);
                \draw [line width=0.4pt] (1.7554294096888614,-1.1475445977883159)-- (-1.7435373424316125,-0.156996783332377);
                \draw [line width=0.4pt] (-2.599229470653089,-3.179605347001402)-- (1.371331049512637,-0.5643865823359441);
                \draw [line width=0.4pt] (2.50614491728459,-2.2873197464715567)-- (-1.9865621664795587,-1.0154467913723637);
                \draw [line width=0.4pt] (-2.2582018253534843,-1.974974560841276)-- (-2.599229470653089,-3.179605347001402);
                \draw [line width=0.4pt] (2.2821549054241452,-1.9472465143500735)-- (2.50614491728459,-2.2873197464715567);
                \draw [line width=0.4pt] (-0.4319460670697387,-0.5283046474273513)-- (-0.8878452142101361,2.8656117803366454);
                \draw [line width=0.4pt] (-0.4319460670697387,-0.5283046474273513)-- (-2.599229470653089,-3.179605347001402);
                \draw [line width=0.4pt] (-0.4319460670697387,-0.5283046474273513)-- (2.50614491728459,-2.2873197464715567);
                \draw [line width=0.4pt] (-1.3067481210192147,2.295208991141382)-- (0.45824590115386454,-3.4014908178028183);
                \draw [line width=0.4pt] (-0.8878452142101361,2.8656117803366454)-- (0.014012755386041872,-2.294532658618477);
                \begin{scriptsize}
                    \draw [fill=black] (0,0) circle (0.6pt);
                    \draw[color=black] (0.1203930648383498-0.15,0.23968091877100078) node {$O$};
                    \draw [fill=black] (-2.2582018253534843,-1.974974560841276) circle (0.6pt);
                    \draw[color=black] (-2.1428204482925035-0.4,-1.738757384098364-0.3) node {$A$};
                    \draw [fill=black] (2.2821549054241452,-1.9472465143500735) circle (0.6pt);
                    \draw[color=black] (2.398594746930335+0.15,-1.7087810461761008-0.3) node {$B$};
                    \draw [fill=black] (-0.8878452142101361,2.8656117803366454) circle (0.6pt);
                    \draw[color=black] (-0.7639089038684075,3.102421190347127) node {$P$};
                    \draw [fill=black] (0.01197654003533044,-1.9611105375956748) circle (0.6pt);
                    \draw[color=black] (0.13538123379948128,-1.7237692151372324) node {$I$};
                    \draw [fill=black] (-2.599229470653089,-3.179605347001402) circle (0.6pt);
                    \draw[color=black] (-2.472560165437396-0.3,-2.9527990699500197-0.3) node {$M$};
                    \draw [fill=black] (2.50614491728459,-2.2873197464715567) circle (0.6pt);
                    \draw[color=black] (2.623417281347307+0.075,-2.0535089322821265-0.3) node {$N$};
                    \draw [fill=black] (0.028025510772083744,-4.589065317236954) circle (0.6pt);
                    \draw[color=black] (0.15036940276061278-0.15,-4.361686952296385-0.5) node {$T$};
                    \draw [fill=black] (0.014012755386041872,-2.294532658618477) circle (0.6pt);
                    \draw[color=black] (0.13538123379948128-0.3,-2.068497101243258-0.3) node {$Q$};
                    \draw [fill=black] (1.7554294096888614,-1.1475445977883159) circle (0.6pt);
                    \draw[color=black] (1.874008833290733,-0.9144080912361285) node {$C$};
                    \draw [fill=black] (-2.1733627213725577,-1.675292708257512) circle (0.6pt);
                    \draw[color=black] (-2.0528914345257143-0.25,-1.4389940048757328) node {$D$};
                    \draw [fill=black] (-0.11703763343915852,-1.5447040182816087) circle (0.6pt);
                    \draw[color=black] (0.0004877131492979636,-1.3190886531866806) node {$H$};
                    \draw [fill=black] (-0.4319460670697387,-0.5283046474273513) circle (0.6pt);
                    \draw[color=black] (-0.31426383503446315,-0.299893163829735) node {$J$};
                    \draw [fill=black] (0.8091498515372272,0.2891460169325444) circle (0.6pt);
                    \draw[color=black] (0.9297541887394497,0.5244561290325003) node {$K$};
                    \draw [fill=black] (-1.7435373424316125,-0.156996783332377) circle (0.6pt);
                    \draw[color=black] (-1.618234534652902-0.25,0.07481106019855374) node {$L$};
                    \draw [fill=black] (1.371331049512637,-0.5643865823359441) circle (0.6pt);
                    \draw[color=black] (1.4843164403013145,-0.32986950175199814) node {$U$};
                    \draw [fill=black] (-1.9865621664795587,-1.0154467913723637) circle (0.6pt);
                    \draw[color=black] (-1.873033406992137-0.25,-0.7795145705859446) node {$V$};
                \end{scriptsize}
            \end{tikzpicture}
        \end{center}

        \begin{solution}
            \hfill
            \begin{enumerate}
                \item[(a)] Gọi \(T\) là giao điểm của hai tiếp tuyến tại \(A\), \(B\) với đường tròn \((O)\); \(C\) là giao của \(MI\) và \(PB\), \(D\) là giao của \(NI\) và \(PA\); \(Q\) là trung điểm của đoạn thẳng \(OT\).\\
                Ta có \(\angle CMD = \angle IMP = \angle INP = \angle CND\). Suy ra \(C\), \(D\), \(M\), \(N\) đồng viên hay \(CD\) đối song \(MN\) ứng với góc \(MPN\).\\
                Ta cũng có \(\angle IQB = 180 \degree - 2 \angle TAB = 180 \degree - 2 \angle APB = \angle ICB\). Suy ra \(B\), \(C\), \(I\), \(Q\) đồng viên. Tương tự, \(A\), \(D\), \(I\), \(Q\) đồng viên. Suy ra \(QC \perp PB\) và \(QD \perp PA\). Suy ra  \(C\), \(D\), \(P\), \(Q\) cùng thuộc đường tròn đường kính \(PQ\).\\
                Tức là \(PQ\) đi qua tâm của \((PCD)\). Mà \(CD\) đối song \(MN\) ứng với góc \(MPN\) nên \(PQ \perp MN\). Vì vậy, đường cao kẻ từ \(P\) của tam giác \(PMN\) luôn đi qua một điểm cố định.
                \item[(b)] Gọi \(K\), \(L\) lần lượt là hình chiếu của \(D\) lên \(PB\) và của \(C\) lên \(PA\); \(U\), \(V\) lần lượt là hình chiếu của \(M\) lên \(PB\) và của \(N\) lên \(PA\); \(J\) là giao của \(DK\) và \(CL\); \(H\) là giao của \(MU\) và \(NV\).\\
                Ta có \(\{L; U\} \in (MC)\) và \(\{K; V\} \in (ND)\). Theo tính chất các đường cao trong tam giác,
                \[\left\{\begin{array}{lcl}
                    \mathcal{P}_{J/(MC)} = \overline{JC} \cdot \overline{JL} = \overline{JD} \cdot \overline{JK} = \mathcal{P}_{J/(ND)} \\
                    \mathcal{P}_{H/(MC)} = \overline{HM} \cdot \overline{HU} = \overline{HN} \cdot \overline{HV} = \mathcal{P}_{H/(ND)} \\
                    \mathcal{P}_{I/(MC)} = \overline{IM} \cdot \overline{IC} = \overline{IN} \cdot \overline{ID} = \mathcal{P}_{I/(ND)} \\
                \end{array}\right.\]
                Do đó \(J\), \(H\), \(I\) thẳng hàng, cùng nằm trên trục đẳng phương của hai đường tròn \((MC)\) và \((ND)\).\\
                Ta cũng có \(\angle LJD = \angle DPC = \angle CMD\), suy ra \(J\) thuộc \((CDMN)\). Khi đó \(\angle JPM = \angle JCD = \angle JMP\) và \(\angle JPN = \angle JDC = \angle JNP\). Suy ra \(JM = JN = JP\) hay \(J\) là tâm đường tròn \((PMN)\). Như vậy \(JH\) là đường thẳng Euler của tam giác \(PMN\). Mà \(J\), \(H\), \(I\) thẳng hàng và \(I\) cố định (do \(I\) là trung điểm \(AB\)) nên đường thẳng Euler của tam giác \(PMN\) luôn đi qua một điểm cố định.
            \end{enumerate}
        \end{solution}

        \begin{problem}
            Cho tam giác \(ABC\) là tam giác nhọn không cân nội tiếp đường tròn \((O)\) bán kính \(R\). Một đường thẳng \(\Delta\) thay đổi sao cho \(\Delta\) vuông góc với \(OA\) và luôn cắt tia \(AB\), \(AC\). Gọi \(M\), \(N\) lần lượt là giao điểm của \(\Delta\) và \(AB\), \(AC\). Giả sử \(BN\) và \(CM\) cắt nhau tại \(K\), \(AK\) cắt \(BC\) tại \(P\).
            \begin{enumerate}
                \item[(a)] Chứng minh rằng đường tròn ngoại tiếp tam giác \(MNP\) luôn đi qua một điểm cố định.
                \item[(b)] Gọi \(H\) là trực tâm của tam giác \(AMN\). Đặt \(BC = a\) và \(\ell\) là khoảng cách từ \(A\) đến \(KH\). Chứng minh rằng \(KH\) đi qua trực tâm tam giác \(ABC\). Từ đó, hãy chứng minh \(\ell \leq \sqrt{4R^2 - a^2}\).
            \end{enumerate}
        \end{problem}

        \begin{center}
            \begin{tikzpicture}[line cap=round,line join=round,>=triangle 45,x=1cm,y=1cm]
                \draw [line width=0.4pt] (5.135520295548944,5.851787341041427)-- (3.046280941758943,0.2531886233280752);
                \draw [line width=0.4pt] (3.046280941758943,0.2531886233280752)-- (10.63,0.36);
                \draw [line width=0.4pt] (10.63,0.36)-- (5.135520295548944,5.851787341041427);
                \draw [line width=0.4pt] (6.813777551152656,2.0363870023262103) circle (4.168192313300178cm);
                \draw [line width=0.4pt] (5.135520295548944,5.851787341041427)-- (6.813777551152656,2.0363870023262103);
                \draw [line width=0.4pt] (3.046280941758943,0.2531886233280752)-- (7.705556855192145,3.2830101312851663);
                \draw [line width=0.4pt] (10.63,0.36)-- (3.483973859479582,1.4260877473359093);
                \draw [line width=0.4pt,dash pattern=on 3pt off 3pt] (5.6305850317989705,2.27311541794611) circle (2.3076818238019214cm);
                \draw [line width=0.4pt] (5.135520295548944,5.851787341041427)-- (4.478891738675122,0.27336594401711073);
                \draw [line width=0.4pt] (0.7437433467797279,0.22075899236866636)-- (7.705556855192145,3.2830101312851663);
                \draw [line width=0.4pt] (0.7437433467797279,0.22075899236866636)-- (3.046280941758943,0.2531886233280752);
                \draw [line width=0.4pt] (3.483973859479582,1.4260877473359093)-- (6.52219187028168,4.465795252631291);
                \draw [line width=0.4pt] (7.705556855192145,3.2830101312851663)-- (4.60824765454407,4.438838687523035);
                \draw [line width=0.4pt] (5.135520295548944,5.851787341041427)-- (5.213942647529942,0.2837186217594931);
                \draw [line width=0.4pt] (3.046280941758943,0.2531886233280752)-- (6.889687639222661,4.09847956118895);
                \draw [line width=0.4pt] (10.63,0.36)-- (4.008264415143962,2.8310452153861982);
                \draw [line width=0.4pt] (5.375918898475544,1.7680993773066207) circle (2.7788788390094568cm);
                \draw [line width=0.4pt] (7.056986929739791,0.8930438736679547) circle (3.612555628845766cm);
                \draw [line width=0.4pt,dash pattern=on 3pt off 3pt] (2.284238216859842,-3.1789996170185595)-- (7.753222116107281,7.327458497117362);
                \draw [line width=0.4pt] (3.046280941758943,0.2531886233280752)-- (10.581274160546368,3.8195853813243454);
                \draw [line width=0.4pt] (10.581274160546368,3.8195853813243454)-- (10.63,0.36);
                \begin{scriptsize}
                    \draw [fill=black] (5.135520295548944,5.851787341041427) circle (0.6pt);
                    \draw[color=black] (5.268698388800859,6.141376097522739) node {$A$};
                    \draw [fill=black] (3.046280941758943,0.2531886233280752) circle (0.6pt);
                    \draw[color=black] (3.1730915009482037-0.2,0.5475907827034616-0.5) node {$B$};
                    \draw [fill=black] (10.63,0.36) circle (0.6pt);
                    \draw[color=black] (10.763478653800341+0.05,0.6465958325232719-0.5) node {$C$};
                    \draw [fill=black] (6.813777551152656,2.0363870023262103) circle (0.6pt);
                    \draw[color=black] (6.951784235737637,2.3296816794600455) node {$O$};
                    \draw [fill=black] (6.498897730920396,2.7522442219254675) circle (0.6pt);
                    \draw[color=black] (6.638268244641571,3.039217869835353) node {$X$};
                    \draw [fill=black] (3.483973859479582,1.4260877473359093) circle (0.6pt);
                    \draw[color=black] (3.618614225137351-0.25,1.7191505389045494) node {$M$};
                    \draw [fill=black] (7.705556855192145,3.2830101312851663) circle (0.6pt);
                    \draw[color=black] (7.842829684115932,3.567244802207674) node {$N$};
                    \draw [fill=black] (4.595065761824061,1.2603282832700065) circle (0.6pt);
                    \draw[color=black] (4.724170614791901-0.225,1.554142122538199) node {$K$};
                    \draw [fill=black] (4.478891738675122,0.27336594401711073) circle (0.6pt);
                    \draw[color=black] (4.608664723335456-0.2,0.5640916243400967-0.5) node {$P$};
                    \draw [fill=black] (6.838140470879472,0.3065943116640376) circle (0.6pt);
                    \draw[color=black] (6.968285077374273,0.5970933076133669-0.5) node {$Q$};
                    \draw [fill=black] (0.7437433467797279,0.22075899236866636) circle (0.6pt);
                    \draw[color=black] (0.8794745134559269-0.15,0.5145890994301916-0.5) node {$T$};
                    \draw [fill=black] (5.982622025508704,3.925960882301006) circle (0.6pt);
                    \draw[color=black] (6.110241312269248,4.21077762603644) node {$H$};
                    \draw [fill=black] (6.52219187028168,4.465795252631291) circle (0.6pt);
                    \draw[color=black] (6.654769086278206,4.755305400045397) node {$U$};
                    \draw [fill=black] (4.60824765454407,4.438838687523035) circle (0.6pt);
                    \draw[color=black] (4.740671456428537-0.135,4.722303716772126) node {$V$};
                    \draw [fill=black] (5.213942647529942,0.2837186217594931) circle (0.6pt);
                    \draw[color=black] (5.351202596984034,0.5805924659767318-0.5) node {$D$};
                    \draw [fill=black] (6.889687639222661,4.09847956118895) circle (0.6pt);
                    \draw[color=black] (7.017787602284178,4.392286884039425) node {$E$};
                    \draw [fill=black] (4.008264415143962,2.8310452153861982) circle (0.6pt);
                    \draw[color=black] (4.1466411575096735-0.25,3.1217220780185277) node {$F$};
                    \draw [fill=black] (5.184246135002579,2.3922019597170823) circle (0.6pt);
                    \draw[color=black] (5.318200913710764-0.25,2.6761993538293813) node {$J$};
                    \draw [fill=black] (10.581274160546368,3.8195853813243454) circle (0.6pt);
                    \draw[color=black] (10.713976128890437,4.11177257621663) node {$L$};
                \end{scriptsize}
            \end{tikzpicture}
        \end{center}

        \begin{solution}
            \hfill
            \begin{enumerate}
                \item[(a)] Gọi \(T\) là giao điểm của hai đường thẳng \(\Delta\) và \(BC\); \(Q\) là trung điểm của đoạn thẳng \(BC\); \(J\) là trực tâm của tam giác \(ABC\).\\
                Không khó để thấy rằng \(J\) và \(O\) chính là hai điểm liên hợp đẳng giác ứng với tam giác \(ABC\); nghĩa là hai đường thẳng nối từ \(A\) đến \(J\) và đến \(O\) sẽ đối xứng qua đường phân giác của góc \(A\), và tương tự cho các đỉnh \(B\), \(C\). Ngoài ra, do \(AJ\) là đường cao của tam giác \(ABC\) và đường thẳng \(\Delta\) vuông góc với \(AO\) nên \(BC\) và \(\Delta\) là hai đường đối song. Nói cách khác, nếu \(\Delta\) cắt \(AB\), \(AC\) lần lượt tại \(M\), \(N\) thì \(B\), \(C\), \(M\), \(N\) đồng viên. Tức là khi này ta sẽ có \(\overline{TM} \cdot \overline{TN} = \overline{TB} \cdot \overline{TC}\).\\
                Mặt khác, theo tính chất của tứ giác toàn phần \(BMNC.AT\) ta thu được \((B,C;T,P) = -1\). Áp dụng hệ thức Maclaurin, ta thu được \(\overline{TB} \cdot \overline{TC} = \overline{TP} \cdot \overline{TQ}\). Từ đó \(\overline{TM} \cdot \overline{TN} = \overline{TP} \cdot \overline{TQ}\), hay \(M\), \(N\), \(P\), \(Q\) đồng viên. Nói cách khác, đường tròn \((MNP)\) đi qua một điểm cố định; đó chính là trung điểm \(Q\) của \(BC\).
                \item[(b)] Gọi \(D\), \(E\), \(F\) lần lượt là hình chiếu vuông góc của \(A\), \(B\), \(C\) lên cạnh đối diện tương ứng trogn tam giác \(ABC\); \(U\), \(V\) lần lượt là hình chiếu vuông góc của \(M\), \(N\) lên \(AN\), \(AM\); \(L\) là điểm đối xứng của \(B\) qua tâm \((O)\).\\
                Ta có \(\angle BEN = \angle BVN = 90 \degree\) nên \(B\), \(V\), \(E\), \(N\) cùng thuộc đường tròn đường kính \(BN\); và \(\angle CFM = \angle CUM = 90 \degree\) nên \(C\), \(U\), \(E\), \(N\) cùng thuộc đường tròn đường kính \(CM\).\\
                Theo tính chất quen thuộc của trực tâm trong tam giác, ta có \(\mathcal{P}_{H/(BN)} = \overline{HV} \cdot \overline{HN} = \overline{HU} \cdot \overline{HM} = \mathcal{P}_{H/(CM)}\) và \(\mathcal{P}_{J/(BN)} = \overline{JB} \cdot \overline{JE} = \overline{JC} \cdot \overline{JF} = \mathcal{P}_{J/(CM)}\). Ngoài ra, do \(B\), \(C\), \(M\), \(N\) đồng viên nên \(\mathcal{P}_{K/(BN)} = \mathcal{P}_{K/(CM)}\). Như vậy \(H\), \(J\), \(K\) cùng thuộc trục đẳng phương của hai đường tròn \((BN)\) và \((CM)\); hay \(H\), \(J\), \(K\) thẳng hàng.\\
                Từ đó \(\ell \leq AJ = CL\) theo bất đẳng thức tam giác mở rộng và một tính chất quen thuộc (tứ giác \(AJLC\) là hình bình hành). Nhưng theo định lí Pythagore, \(CL = \sqrt{BL^2 - BC^2} = \sqrt{4R^2 - a^2}\) nên vì vậy ta có điều cần phải chứng minh.
            \end{enumerate}
            Chứng minh hoàn tất.
        \end{solution}

        \begin{problem}
            Cho tam giác \(ABC\) nhọn nội tiếp đường tròn \((O)\). Đường tròn tâm \(I\) đi qua hai điểm \(B\) và \(C\) lần lượt cắt các tia \(BA\), \(CA\) tại \(E\) và \(F\).
            \begin{enumerate}
                \item[(a)] Giả sử các tia \(BF\), \(CE\) cắt nhau tại \(D\) và \(T\) là tâm đường tròn \((AEF)\). Chứng minh rằng \(OT \parallel ID\).
                \item[(b)] Trên \(BF\), \(CE\) lần lượt lấy các điểm \(G\), \(H\) sao cho \(AG \perp CE\) và \(AH \perp BF\). Các đường tròn \((ABF)\) và \((ACE)\) cắt \(BC\) tại các điểm \(M\) và \(N\) (khác \(B\) và \(C\)) và cắt \(EF\) tại các điểm \(P\) và \(Q\) (khác \(E\) và \(F\)). Gọi \(K\) là giao điểm của \(MP\) và \(NQ\). Chứng minh rằng \(DK \perp GH\).
            \end{enumerate}
        \end{problem}

        \begin{center}
            \begin{tikzpicture}[line cap=round,line join=round,>=triangle 45,x=1cm,y=1cm,scale=0.8]
                \draw [line width=0.4pt] (0,6)-- (-2,0);
                \draw [line width=0.4pt] (-2,0)-- (5,0);
                \draw [line width=0.4pt] (5,0)-- (0,6);
                \draw [line width=0.4pt] (1.5,2.1666666666666665) circle (4.1163630117428225cm);
                \draw [line width=0.4pt] (1.5,0.37964778867924776) circle (3.5205301367051303cm);
                \draw [line width=0.4pt] (0,4.212981122012581) circle (1.7870188779874194cm);
                \draw [line width=0.4pt] (0,6)-- (-8.185257821383653,0);
                \draw [line width=0.4pt] (-8.185257821383653,0)-- (1.757723486545002,3.8907318161459976);
                \draw [line width=0.4pt] (-8.185257821383653,0)-- (-2,0);
                \draw [line width=0.4pt] (-2,0)-- (1.757723486545002,3.8907318161459976);
                \draw [line width=0.4pt] (5,0)-- (-1.0722113267924513,2.7833660196226457);
                \draw [line width=0.4pt] (-1.3254123902964983,3.1084707967654985) circle (3.1808268009291907cm);
                \draw [line width=0.4pt] (2.8254123902964983,3.2711769919137486) circle (3.9280313370315505cm);
                \draw [line width=0.4pt] (1.757723486545002,3.8907318161459976)-- (6.019001189942999,5.558188308779998);
                \draw [line width=0.4pt] (0,6)-- (-3.427996522419917,-1.478541867973732);
                \draw [line width=0.4pt] (0,6)-- (7.307539841371087,-1.0577247130808296);
                \draw [line width=0.4pt] (-3.427996522419917,-1.478541867973732)-- (7.307539841371087,-1.0577247130808296);
                \draw [line width=0.4pt] (0.2514020959660032,-0.4135606445586401)-- (-4.120659824474398,1.5904948683557965);
                \draw [line width=0.4pt] (0.2514020959660032,-0.4135606445586401)-- (6.019001189942999,5.558188308779998);
                \draw [line width=0.4pt] (-2,0)-- (-3.427996522419917,-1.478541867973732);
                \draw [line width=0.4pt] (5,0)-- (7.307539841371087,-1.0577247130808296);
                \draw [line width=0.4pt] (0,6)-- (0.28743892941934707,-1.3329022908674881);
                \begin{scriptsize}
                    \draw [fill=black] (0,6) circle (0.6pt);
                    \draw[color=black] (0.16697072373364177-0.1,6.300314517013739) node {$A$};
                    \draw [fill=black] (-2,0) circle (0.6pt);
                    \draw[color=black] (-1.8382389985496392-0.2,0.3045389117706615-0.5) node {$B$};
                    \draw [fill=black] (5,0) circle (0.6pt);
                    \draw[color=black] (5.1502146870316965,0.3045389117706615-0.5) node {$C$};
                    \draw [fill=black] (1.5,2.1666666666666665) circle (0.6pt);
                    \draw[color=black] (1.6559878442410285-0.1,2.468577126908063) node {$O$};
                    \draw [fill=black] (1.5,0.37964778867924776) circle (0.6pt);
                    \draw[color=black] (1.6559878442410285-0.1,0.6817565822991994) node {$I$};
                    \draw [fill=black] (-1.0722113267924513,2.7833660196226457) circle (0.6pt);
                    \draw[color=black] (-0.9051216030316768-0.3,3.084037536717783) node {$E$};
                    \draw [fill=black] (1.757723486545002,3.8907318161459976) circle (0.6pt);
                    \draw[color=black] (1.9140841451289756,4.195836986696632) node {$F$};
                    \draw [fill=black] (0.14801193482633823,2.2240429361592597) circle (0.6pt);
                    \draw[color=black] (0.3059456549809979-0.05,2.528137811728359) node {$D$};
                    \draw [fill=black] (0,4.212981122012581) circle (0.6pt);
                    \draw[color=black] (0.16697072373364177,4.513493972404874) node {$T$};
                    \draw [fill=black] (-1.7041667177527062,4.750802903263005) circle (0.6pt);
                    \draw[color=black] (-1.5404355744481617-0.3,5.049540135787534) node {$R$};
                    \draw [fill=black] (-8.185257821383653,0) circle (0.6pt);
                    \draw[color=black] (-8.032550219860369-0.2,0.3045389117706615) node {$S$};
                    \draw [fill=black] (-3.427996522419917,-1.478541867973732) circle (0.6pt);
                    \draw[color=black] (-3.267695434236731-0.3,-1.1646246471299597) node {$G$};
                    \draw [fill=black] (7.307539841371087,-1.0577247130808296) circle (0.6pt);
                    \draw[color=black] (7.47308139502322,-0.7476998533878917) node {$H$};
                    \draw [fill=black] (-0.6508247805929875,0) circle (0.6pt);
                    \draw[color=black] (-0.4881968092896084,0.3045389117706615) node {$M$};
                    \draw [fill=black] (0.6508247805929964,0) circle (0.6pt);
                    \draw[color=black] (0.8022846951501268,0.3045389117706615) node {$N$};
                    \draw [fill=black] (-4.120659824474398,1.5904948683557965) circle (0.6pt);
                    \draw[color=black] (-3.962570090473511,1.8928238403118738) node {$P$};
                    \draw [fill=black] (6.019001189942999,5.558188308779998) circle (0.6pt);
                    \draw[color=black] (6.182599890583485,5.863536161664905) node {$Q$};
                    \draw [fill=black] (0.2514020959660032,-0.4135606445586401) circle (0.6pt);
                    \draw[color=black] (0.40521346301482364,-0.11238588197140675-0.5) node {$K$};
                    \draw [fill=black] (0.28743892941934707,-1.3329022908674881) circle (0.6pt);
                \end{scriptsize}
            \end{tikzpicture}
        \end{center}

        \begin{solution}
            \hfill
            \begin{enumerate}
                \item[(a)] Gọi \(R\) là giao điểm khác \(A\) của hai đường tròn \((O)\) và \((AEF)\).\\
                Theo tính chất tâm đẳng phương, trục đẳng phương của các cặp đường tròn \((O)\) và \((AEF)\), \((O)\) và \((I)\), \((AEF)\) và \((I)\) đồng quy. Nói cách khác, \(AR\), \(EF\), \(BC\) đồng quy. Gọi \(S\) là điểm đồng quy.\\
                Áp dụng định lí Brocard cho tứ giác toàn phần \(BEFC.AS\) có \(D\) là giao của \(BF\) và \(EC\), \(I\) là tâm ngoại tiếp tứ giác \(BEFC\), ta được \(ID \perp AS\) hay \(ID \perp AR\). Mặt khác, do \(AR\) là trục đẳng phương của \((O)\) và \((AEF)\) nên \(OT \perp AR\). Từ đó \(ID \parallel OT\).
                \item[(b)] Ta có \(\angle BMP = \angle BFP = \angle BFE = \angle BCE = \angle NCE = \angle NQE\), do đó \(M\), \(N\), \(P\), \(Q\) đồng viên. Khi đó \(\mathcal{P}_{K/(ABF)} = \overline{KM} \cdot \overline{KP} = \overline{KN} \cdot \overline{KQ} = \mathcal{P}_{K/(ACE)}\). Lại có \(\mathcal{P}_{D/(ABF)} = \overline{DB} \cdot \overline{DF} = \overline{DC} \cdot \overline{DE} = \mathcal{P}_{D/(ACE)}\) nên \(D\) và \(K\) nằm trên trục đẳng phương của hai đường tròn \((ABF)\) và \((ACE)\). Suy ra \(A\), \(D\), \(K\) thẳng hàng. Mà theo giả thiết, \(D\) là trực tâm của tam giác \(AGH\) nên \(DK \perp GH\).
            \end{enumerate}
        \end{solution}

        \begin{problem}
            Cho tam giác \(ABC\) nhọn, các đường cao \(AD\), \(BE\), \(CF\), trọng tâm \(G\) và trực tâm \(H\).
            \begin{enumerate}
                \item[(a)] Đường tròn \((BHC)\) cắt đường tròn đường kính \(AH\) tại \(T\) khác \(H\). Chứng minh rằng \(A\), \(T\), \(G\) thẳng hàng.
                \item[(b)] Các điểm \(I\), \(J\), \(K\) lần lượt nằm trên các đường thẳng \(BC\), \(CA\), \(AB\) sao cho \(HI\), \(HJ\), \(HK\) tương ứng vuông góc với \(AG\), \(BG\), \(CG\). Chứng minh rằng các đường tròn \((AGD)\), \((BGE)\), \((CGF)\) cùng đi qua một điểm \(L\) khác \(G\) và \(I\), \(J\), \(K\), \(L\) thẳng hàng.
            \end{enumerate}
        \end{problem}

        \begin{center}
            \begin{tikzpicture}[line cap=round,line join=round,>=triangle 45,x=1cm,y=1cm,scale=0.8]
                \draw [line width=0.4pt] (0,4)-- (-2,0);
                \draw [line width=0.4pt] (-2,0)-- (5,0);
                \draw [line width=0.4pt] (5,0)-- (0,4);
                \draw [line width=0.4pt] (0,4)-- (0,0);
                \draw [line width=0.4pt] (-2,0)-- (0.7317073170731707,3.4146341463414633);
                \draw [line width=0.4pt] (5,0)-- (-0.6,2.8);
                \draw [line width=0.4pt] (0,3.25) circle (0.75cm);
                \draw [line width=0.4pt] (0,2.5)-- (0.49315068493150965,2.684931506849315);
                \draw [line width=0.4pt] (0,4)-- (1.5,0);
                \draw [line width=0.4pt] (0.49315068493150965,2.684931506849315)-- (5,0);
                \draw [line width=0.4pt] (0,2.5)-- (-6.666666666666667,0);
                \draw [line width=0.4pt] (0,2.5)-- (-1.0344827586206897,4.827586206896552);
                \draw [line width=0.4pt] (0,2.5)-- (1.5,7);
                \draw [line width=0.4pt] (-1.2777777777777777,2) circle (2.373334373699314cm);
                \draw [line width=0.4pt] (-1.1436781609195403,2.1149425287356323) circle (2.281725003575324cm);
                \draw [line width=0.4pt] (1.5,0.75) circle (3.579455265819088cm);
                \draw [line width=0.4pt] (0.75,1.625) circle (1.789727632909544cm);
                \draw [line width=0.4pt,dash pattern=on 3pt off 3pt] (-7.731187612385848,-0.9124465249021549)-- (2.510415885383906,7.86607075890049);
                \draw [line width=0.4pt] (-6.666666666666667,0)-- (-2,0);
                \draw [line width=0.4pt] (-1.0344827586206897,4.827586206896552)-- (0,4);
                \draw [line width=0.4pt] (1.5,7)-- (0,4);
                \draw [line width=0.4pt] (-6.666666666666667,0)-- (0.7317073170731707,3.4146341463414633);
                \draw [shift={(1.5,-0.75)},line width=0.4pt]  plot[domain=-0.3440084895511619:3.441668597497034,variable=\t]({1*3.579455265819088*cos(\t r)+0*3.579455265819088*sin(\t r)},{0*3.579455265819088*cos(\t r)+1*3.579455265819088*sin(\t r)});
                \draw [shift={(5.333333333333333,7.666666666666667)},line width=0.4pt]  plot[domain=3.285429366430654:4.8967788739780715,variable=\t]({1*7.673909622147559*cos(\t r)+0*7.673909622147559*sin(\t r)},{0*7.673909622147559*cos(\t r)+1*7.673909622147559*sin(\t r)});
                \begin{scriptsize}
                    \draw [fill=black] (0,4) circle (0.6pt);
                    \draw[color=black] (0.173021817894646-0.2,4.335631907022558) node {$A$};
                    \draw [fill=black] (-2,0) circle (0.6pt);
                    \draw[color=black] (-1.8267653626800477-0.4,0.33605754587321446-0.5) node {$B$};
                    \draw [fill=black] (5,0) circle (0.6pt);
                    \draw[color=black] (5.161738225349796+0.1,0.33605754587321446-0.5) node {$C$};
                    \draw [fill=black] (0,0) circle (0.6pt);
                    \draw[color=black] (0.173021817894646-0.1,0.33605754587321446-0.5) node {$D$};
                    \draw [fill=black] (0.7317073170731707,3.4146341463414633) circle (0.6pt);
                    \draw[color=black] (0.9041268086423835,3.755048532017008) node {$E$};
                    \draw [fill=black] (-0.6,2.8) circle (0.6pt);
                    \draw[color=black] (-0.42906464507407904-0.4,3.1314589810851214-0.3) node {$F$};
                    \draw [fill=black] (1.5,0) circle (0.6pt);
                    \draw[color=black] (1.6782379753164585-0.1,0.33605754587321446-0.5) node {$M$};
                    \draw [fill=black] (1,1.3333333333333333) circle (0.6pt);
                    \draw[color=black] (1.1621638642004084+0.1,1.6692489995896622-0.2) node {$G$};
                    \draw [fill=black] (0,2.5) circle (0.6pt);
                    \draw[color=black] (0.173021817894646,2.8304157496007623-0.6) node {$H$};
                    \draw [fill=black] (0.49315068493150965,2.684931506849315) circle (0.6pt);
                    \draw[color=black] (0.6675928410475273-0.1,3.023943541269279) node {$T$};
                    \draw [fill=black] (-6.666666666666667,0) circle (0.6pt);
                    \draw[color=black] (-6.492935450687666-0.1,0.33605754587321446-0.5) node {$I$};
                    \draw [fill=black] (-1.0344827586206897,4.827586206896552) circle (0.6pt);
                    \draw[color=black] (-0.859126404337454,5.152749249622961) node {$J$};
                    \draw [fill=black] (1.5,7) circle (0.6pt);
                    \draw[color=black] (1.6782379753164585,7.324561133902981) node {$K$};
                    \draw [fill=black] (-1.5882352941176539,4.352941176470586) circle (0.6pt);
                    \draw[color=black] (-1.4182066913798417,4.679681314433254) node {$L$};
                    \draw [fill=black] (1.5,0.75) circle (0.6pt);
                    \draw[color=black] (1.6782379753164585,1.0886656245841124) node {$O$};
                \end{scriptsize}
            \end{tikzpicture}
        \end{center}

        \begin{solution}
            \hfill
            \begin{enumerate}
                \item[(a)] Gọi \(M\) là trung điểm đoạn thẳng \(BC\) và \(T'\) là hình chiếu vuông góc của \(H\) lên \(AM\).\\
                Do \(\angle AT'H = 90 \degree\) nên \(T'\) thuộc đường tròn đường kính \(AH\). Không khó để chỉ ra rằng \(M\) là tâm đường tròn ngoại tiếp tứ giác \(BFEC\) và \(ME\), \(MF\) là tiếp tuyến tại \(E\), \(F\) của đường tròn đường kính \(AH\). Khi đó \(MC^2 = ME^2 = \overline{MT'} \cdot \overline{MA}\), suy ra \(\angle T'CB = \angle T'AC = \angle T'AE = \angle THE\). Do đó \(T'\) thuộc \((M)\), \(T' \equiv T\). Mà trọng tâm \(G\) nằm trên trung tuyến \(AM\) nên \(A\), \(T\), \(G\) thẳng hàng.
                \item[(b)] Ta có \(\overline{HA} \cdot \overline{HD} = \overline{HB} \cdot \overline{HE} = \overline{HC} \cdot \overline{HF} < 0\), hay \(\mathcal{P}_{H/(AGD)} = \mathcal{P}_{H/(BGE)} = \mathcal{P}_{H/(CGF)} < 0\). Mà \(G\) lại thuộc ba đường tròn \((AGD)\), \((BGE)\), \((CGF)\) nên ba đường tròn \((AGD)\), \((BGE)\), \((CGF)\) đồng trục, giao nhau tại hai điểm, đó là điểm \(G\) và một điểm \(L\) khác \(G\).\\
                Gọi \((O)\) là tâm đường tròn ngoại tiếp tam giác \(ABC\). Ta chứng minh các điểm \(I\), \(J\), \(K\), \(L\) cùng thuộc trục đẳng phương của \((O)\) và đường tròn Euler của tam giác \(ABC\), đi qua chân ba đường cao và có tâm là trung điểm đoạn thẳng \(OH\).\\
                Do vai trò của \(I\), \(J\), \(K\) trong mô hình là như nhau, ta chỉ cần chỉ ra \(I\) thuộc trục đẳng phương của \((O)\) và đường tròn Euler. Theo tính chất tâm đẳng phương, trục đẳng phương của các cặp đường tròn \((AH)\) và \((BHC)\), \((AH)\) và đường tròn Euler, \((BHC)\) và đường tròn Euler đồng quy. Nói cách khác, \(EF\), \(HT\), \(BC\) đồng quy. Mà \(IH \perp AM\) nên \(I\), \(H\), \(T\) thẳng hàng, suy ra \(E\), \(F\), \(I\) thẳng hàng. Khi đó, \(\mathcal{P}_{I/Euler} = \overline{IE} \cdot \overline{IF} = \overline{IB} \cdot \overline{IC} = \mathcal{P}_{I/(O)}\), hay \(I\) thuộc trục đẳng phương của \((O)\) và đường tròn Euler của tam giác \(ABC\).\\
                Đối với điểm \(L\), nhận thấy rằng \(GH\), ngoài là trục đẳng phương của ba đường tròn \((AGD)\), \((BGE)\), \((CGF)\), còn là đường thẳng Euler của tam giác \(ABC\). Mà \(H\) thuộc trục đẳng phương của ba đường tròn \((AGD)\), \((BGE)\), \((CGF)\) nên ta có \(L\), \(H\), \(G\), \(O\) thẳng hàng. Khi đó, biến đổi góc ta được \(\angle HID = \angle HAM = \angle DAG = \angle DLH\), suy ra \(I\), \(L\), \(H\), \(D\) đồng viên. Mà \(\angle HDI = 90 \degree\) nên \(IL \perp OH\), hay \(IL\) chính là trục đẳng phương của \((O)\) và đường tròn Euler. Tương tự với các đường thẳng \(JL\), \(KL\), ta thu được \(I\), \(J\), \(K\), \(L\) thẳng hàng.
            \end{enumerate}
        \end{solution}

        \begin{problem}
            Cho tam giác không cân \(ABC\) và đường tròn \((I)\) nội tiếp tam giác. Gọi \(D\), \(E\), \(F\) lần lượt là các tiếp điểm của \((I)\) với \(BC\), \(CA\), \(AB\). Đường thẳng qua \(E\) vuông góc \(BI\) cắt \((I)\) tại \(K\) khác \(E\), đường thẳng qua \(F\) vuông góc \(CI\) cắt \((I)\) tại \(L\) khác \(F\). Gọi \(J\) là trung điểm của đoạn thẳng \(KL\).
            \begin{itemize}
                \item[(a)] Chứng minh rằng \(D\), \(I\), \(J\) thẳng hàng.
                \item[(b)] Giả sử các điểm \(B\) và \(C\) cố định, \(A\) thay đổi sao cho tỉ số \(\dfrac{AB}{AC}\) không đổi. Gọi \(M\), \(N\) tương ứng là các giao điểm \(IE\), \(IF\) với \((I)\) (\(M\) khác \(E\), \(N\) khác \(F\)). \(MN\) cắt \(IB\), \(IC\) tại \(P\) và \(Q\). Chứng minh rằng đường trung trực \(PQ\) luôn đi qua một điểm cố định.
            \end{itemize}
        \end{problem}

        \begin{center}
            \begin{tikzpicture}[line cap=round,line join=round,>=triangle 45,x=1cm,y=1cm]
                \draw [line width=0.4pt] (0,7.35)-- (-1.75,0);
                \draw [line width=0.4pt] (-1.75,0)-- (4.2,0);
                \draw [line width=0.4pt] (4.2,0)-- (0,7.35);
                \draw [line width=0.4pt] (0.7700454825047698,1.9904799021438024) circle (1.9904799021438024cm);
                \draw [line width=0.4pt] (2.131036069793949,3.442965654965031)-- (-0.590945137349195,3.4429656244515714);
                \draw [line width=0.4pt,dash pattern=on 3pt off 3pt] (0.7700454825047697,0)-- (0.7700454662223771,3.4429656397083015);
                \draw [line width=0.4pt] (0.7700454825047697,0)-- (2.131036069793949,3.442965654965031);
                \draw [line width=0.4pt] (0.7700454825047697,0)-- (-0.590945137349195,3.4429656244515714);
                \draw [line width=0.4pt] (0.7700454825047697,0)-- (-1.1663057261705982,2.4515159500834875);
                \draw [line width=0.4pt] (0.7700454825047697,0)-- (2.49826600707769,2.9780344876140425);
                \draw [line width=0.4pt] (2.49826600707769,2.9780344876140425)-- (-1.1663057261705982,2.4515159500834875);
                \draw [line width=0.4pt] (4.2,0)-- (1.9140710687354654,2.8940984886889645);
                \draw [line width=0.4pt] (-1.75,0)-- (-0.2510133850586303,2.5830233730335688);
                \draw [line width=0.4pt] (1.225,0) circle (2.975cm);
                \draw [line width=0.4pt] (-0.2004724311075493,7.569284013455018)-- (0.9572264765420142,-0.4883063469580853);
                \draw [line width=0.4pt] (0,7.35)-- (1.0560337853148427,0);
                \draw [line width=0.4pt] (0.8315288521604792,2.738560997608862)-- (1.225,0);
                \draw [line width=0.4pt] (-1.75,0)-- (1.9140710687354654,2.8940984886889645);
                \draw [line width=0.4pt] (4.2,0)-- (-0.2510133850586303,2.5830233730335688);
                \draw [line width=0.4pt] (-1.1663057261705982,2.4515159500834875)-- (2.706396691180138,1.5294438542041175);
                \draw [line width=0.4pt] (2.49826600707769,2.9780344876140425)-- (-0.9581750420681515,1.0029253166735619);
                \draw [line width=0.4pt] (-0.9581750420681515,1.0029253166735619)-- (2.706396691180138,1.5294438542041175);
                \draw [line width=0.4pt] (2.49826600707769,2.9780344876140425)-- (2.131036069793949,3.442965654965031);
                \draw [line width=0.4pt] (-0.590945137349195,3.4429656244515714)-- (-1.1663057261705982,2.4515159500834875);
                \draw [line width=0.4pt] (0.7085621179991716,1.2423988693412291)-- (0.8315288521604792,2.738560997608862);
                \begin{scriptsize}
                    \draw [fill=black] (0,7.35) circle (0.6pt);
                    \draw[color=black] (0.12741006358553514-0.1,7.389901907261489+0.2) node {$A$};
                    \draw [fill=black] (-1.75,0) circle (0.6pt);
                    \draw[color=black] (-1.6313564917046453-0.3,0.24491277639509096-0.5) node {$B$};
                    \draw [fill=black] (4.2,0) circle (0.6pt);
                    \draw[color=black] (4.3201839052148046+0.1,0.24491277639509096-0.5) node {$C$};
                    \draw [fill=black] (0.7700454825047698,1.9904799021438024) circle (0.6pt);
                    \draw[color=black] (0.896870431524989,2.239228423911646) node {$I$};
                    \draw [fill=black] (0.7700454825047697,0) circle (0.6pt);
                    \draw[color=black] (0.896870431524989-0.2,0.24491277639509096-0.5) node {$D$};
                    \draw [fill=black] (2.49826600707769,2.9780344876140425) circle (0.6pt);
                    \draw[color=black] (2.6242304411849875,3.2285346112623783) node {$E$};
                    \draw [fill=black] (-1.1663057261705982,2.4515159500834875) circle (0.6pt);
                    \draw[color=black] (-1.0346321247311911-0.3,2.694623335549285) node {$F$};
                    \draw [fill=black] (2.131036069793949,3.442965654965031) circle (0.6pt);
                    \draw[color=black] (2.263055166437897,3.6839295229000166) node {$K$};
                    \draw [fill=black] (-0.590945137349195,3.4429656244515714) circle (0.6pt);
                    \draw[color=black] (-0.4693143033879188-0.2,3.6839295229000166) node {$L$};
                    \draw [fill=black] (0.7700454662223771,3.4429656397083015) circle (0.6pt);
                    \draw[color=black] (0.896870431524989,3.6839295229000166) node {$J$};
                    \draw [fill=black] (-0.9581750420681515,1.0029253166735619) circle (0.6pt);
                    \draw[color=black] (-0.8304895781350095,1.2499222365609142) node {$M$};
                    \draw [fill=black] (2.706396691180138,1.5294438542041175) circle (0.6pt);
                    \draw[color=black] (2.828372987781169,1.7681302394589165) node {$N$};
                    \draw [fill=black] (-0.3739801140698253,1.086861307428421) circle (0.6pt);
                    \draw[color=black] (-0.2494684839766463,1.3284386006363689) node {$P$};
                    \draw [fill=black] (1.7911043500681683,1.397936431254037) circle (0.6pt);
                    \draw[color=black] (1.9175831645058972,1.6425040569381886) node {$Q$};
                    \draw [fill=black] (1.9140710687354654,2.8940984886889645) circle (0.6pt);
                    \draw[color=black] (2.0432093470266244-0.2,3.1343149743718324) node {$U$};
                    \draw [fill=black] (-0.2510133850586303,2.5830233730335688) circle (0.6pt);
                    \draw[color=black] (-0.12384230145591915,2.8202495180700127) node {$V$};
                    \draw [fill=black] (1.0560337853148427,0) circle (0.6pt);
                    \draw[color=black] (1.179529342196625-0.08,0.24491277639509096-0.5) node {$R$};
                    \draw [fill=black] (0.8870675705682014,0) circle (0.6pt);
                    \draw[color=black] (1.0067933412306251-0.1,0.24491277639509096-0.5) node {$X$};
                    \draw [fill=black] (0.7085621179991716,1.2423988693412291) circle (0.6pt);
                    \draw[color=black] (0.8340573402646254,1.4854713287872787) node {$T$};
                    \draw [fill=black] (1.225,0) circle (0.6pt);
                    \draw[color=black] (1.3522653431626248-0.05,0.24491277639509096-0.5) node {$G$};
                    \draw [fill=black] (0.8315288521604792,2.738560997608862) circle (0.6pt);
                    \draw[color=black] (0.9596835227853526,2.9772822462209225) node {$S$};
                \end{scriptsize}
            \end{tikzpicture}
        \end{center}
        
        \begin{solution}
            \hfill
            \begin{itemize}
                \item[(a)] Do \(J\) là trung điểm của đoạn thẳng \(KL\) và \(K, L \in (I)\) nên \(IJ \perp KL\).\\
                Mặt khác, ta có các cặp điểm \((K;E)\) và \((F;D)\) đối xứng nhau qua \(BI\). Nói cách khác, \(KD\) và \(EF\) đối xứng nhau qua \(BI\). Khi đó \(KD = EF\) và \(\angle IDK = \angle IFE\).\\
                Tương tự, ta có \(LD = EF\) và \(\angle IDL = \angle IEF\). Mà \(\angle IFE = \angle IEF\) nên \(\angle IDK = \angle IDL\). Suy ra tam giác \(DKL\) cân tại \(D\). Từ đó \(DJ \perp KL\) và \(D\), \(I\), \(J\) thẳng hàng.
                \item[(b)] Gọi \(G\) là trung điểm của đoạn thẳng \(BC\); \(U\), \(V\) lần lượt là giao điểm của \(BI\), \(CI\) với \(EF\); \(T\) là trung điểm của đoạn thẳng \(PQ\); \(S\) là giao điểm của \(TI\) và \(EF\); \(R\) là giao điểm của \(BC\) và đường phân giác trong của góc \(A\); \(X\) là giao điểm của \(BC\) và trung trực đoạn thẳng \(PQ\).\\
                Nhận thấy rằng tứ giác \(EFMN\) là hình chữ nhật. Khi đó \(\triangle IEU \sim \triangle IMP\), \(\triangle IFV \sim \triangle INQ\) và \(\triangle IEF \cup \{S\} \sim \triangle IMN \cup \{T\}\). Suy ra \(IU = IP\) và \(IV = IQ\); kết hợp với \(UV \parallel PQ\), ta thu được tứ giác \(UVPQ\) là hình bình hành. Mà \(T\) là trung điểm của đoạn thẳng \(PQ\) nên \(S\) là trung điểm của đoạn thẳng \(UV\). Hơn nữa, ta có \(CU \perp BI\) và \(BV \perp CI\) theo bổ đề Iran. Khi đó tứ giác \(BVUC\) nội tiếp đường tròn có tâm là điểm \(G\). Suy ra \(GS \perp UV\).\\
                Với \(I\) là trung điểm đoạn thẳng \(ST\) và \(TX \parallel IR \parallel SG\) (do cùng vuông góc với \(MN\)), ta thu được \(R\) chính là trung điểm của đoạn thẳng \(XG\). Với các điểm \(B\), \(C\) cố định và tỉ lệ \(\dfrac{AB}{AC}\) không thay đổi, áp dụng tính chất đường phân giác ta có \(\dfrac{AB}{AC} = \dfrac{RB}{RC}\), hay \(R\) là điểm cố định. Mà \(G\) cũng là điểm cố định nên \(X\) là điểm cố định.\\
                Tóm lại, trung trực của đoạn thẳng \(PQ\) luôn đi qua một điểm cố định.
            \end{itemize}
        \end{solution}
        
        \begin{problem}
            Cho tam giác \(ABC\) nhọn, không cân nội tiếp đường tròn \((O)\). Gọi \(H\) là trực tâm của tam giác \(ABC\) và \(E\), \(F\) lần lượt là chân các đường cao hạ từ đỉnh \(B\), \(C\). \(AH\) cắt \((O)\) tại \(D\) (\(D\) khác \(A\)).
            \begin{itemize}
                \item[(a)] Gọi \(I\) là trung điểm của đoạn thẳng \(AH\), \(EI\) cắt \(BD\) tại \(M\) và \(FI\) cắt \(CD\) tại \(N\). Chứng minh rằng \(MN \perp OH\).
                \item[(b)] Các đường thẳng \(DE\), \(DF\) cắt \((O)\) lần lượt tại \(P\), \(Q\) (\(P\) và \(Q\) khác \(D\)). Đường tròn ngoại tiếp tam giác \(AEF\) cắt \((O)\) và \(AO\) lần lượt tại \(R\) và \(S\) (\(R\) và \(S\) khác \(A\)). Chứng minh rằng \(BP\), \(CQ\) và \(RS\) đồng quy.
            \end{itemize}
        \end{problem}

        \begin{center}
            \hspace{-2cm}
            \begin{tikzpicture}[line cap=round,line join=round,>=triangle 45,x=1cm,y=1cm,scale=0.8]
                \draw [line width=0.4pt] (0,6.5)-- (-2,0);
                \draw [line width=0.4pt] (-2,0)-- (5,0);
                \draw [line width=0.4pt] (5,0)-- (0,6.5);
                \draw [line width=0.4pt] (1.5,2.480769230769231) circle (4.290013517033642cm);
                \draw [line width=0.4pt] (0.75,2.0096153846153846) circle (2.145006758516821cm);
                \draw [line width=0.4pt] (0.54585798816568,1.2403846153846156) circle (2.8319510747874586cm);
                \draw [line width=0.4pt] (1.881656804733728,1.240384615384616) circle (3.3559824608520086cm);
                \draw [line width=0.4pt] (2.3977695167286246,3.382899628252788)-- (-11.030534351145043,6.946564885496185);
                \draw [line width=0.4pt] (0,-1.5384615384615379)-- (-11.030534351145043,6.946564885496185);
                \draw [line width=0.4pt] (-4.776859504132232,-3.008264462809916)-- (0,4.019230769230769);
                \draw [line width=0.4pt] (5,0)-- (-4.776859504132232,-3.008264462809916);
                \draw [line width=0.4pt] (0,6.5)-- (0,-1.5384615384615379);
                \draw [line width=0.4pt] (0,4.019230769230769) circle (2.480769230769231cm);
                \draw [line width=0.4pt] (0,-1.5384615384615379)-- (3.7406559605306873,6.139140021554338);
                \draw [line width=0.4pt] (0,-1.5384615384615379)-- (-2.350844526286875,4.3715924405637825);
                \draw [line width=0.4pt] (-11.030534351145043,6.946564885496185)-- (-4.776859504132232,-3.008264462809916);
                \draw [line width=0.4pt] (0,6.5)-- (1.6252989048528081,2.1450324216123473);
                \draw [line width=0.4pt] (-2.479974367190004,4.082024991989744)-- (1.6252989048528081,2.1450324216123473);
                \draw [line width=0.4pt] (-2,0)-- (3.7406559605306873,6.139140021554338);
                \draw [line width=0.4pt] (5,0)-- (-2.350844526286875,4.3715924405637825);
                \draw [line width=0.4pt] (2.3977695167286246,3.382899628252788)-- (-1.3945945945945946,1.9675675675675677);
                \draw [line width=0.4pt] (-2,0)-- (2.3977695167286246,3.382899628252788);
                \draw [line width=0.4pt] (5,0)-- (-1.3945945945945946,1.9675675675675677);
                \draw [line width=0.4pt] (3,-1.5384615384615379)-- (1.5,2.480769230769231);
                \draw [line width=0.4pt] (0,0)-- (2.3977695167286246,3.382899628252788);
                \draw [line width=0.4pt] (0,0)-- (-1.3945945945945946,1.9675675675675677);
                \draw [line width=0.4pt] (-2.479974367190004,4.082024991989744)-- (3,-1.5384615384615379);
                \draw [line width=0.4pt] (-2.479974367190004,4.082024991989744)-- (2.3977695167286246,3.382899628252788);
                \draw [line width=0.4pt] (-2.479974367190004,4.082024991989744)-- (-1.3945945945945946,1.9675675675675677);
                \draw [line width=0.4pt] (3,-1.5384615384615379)-- (5,0);
                \begin{scriptsize}
                    \draw [fill=black] (0,6.5) circle (0.6pt);
                    \draw[color=black] (0.11456130653340307,6.7131327388174835) node {$A$};
                    \draw [fill=black] (-2,0) circle (0.6pt);
                    \draw[color=black] (-1.8888572082472437-0.2,0.21224408881493573-0.5) node {$B$};
                    \draw [fill=black] (5,0) circle (0.6pt);
                    \draw[color=black] (5.102664547415829,0.21224408881493573-0.5) node {$C$};
                    \draw [fill=black] (1.5,2.480769230769231) circle (0.6pt);
                    \draw[color=black] (1.6137180182740227,2.692667011876704) node {$O$};
                    \draw [fill=black] (2.3977695167286246,3.382899628252788) circle (0.6pt);
                    \draw[color=black] (2.5132120453183946+0.1,3.592161038921082-0.1) node {$E$};
                    \draw [fill=black] (-1.3945945945945946,1.9675675675675677) circle (0.6pt);
                    \draw[color=black] (-1.2891945235509956-0.3,2.174776511457214-0.3) node {$F$};
                    \draw [fill=black] (0,1.5384615384615385) circle (0.6pt);
                    \draw[color=black] (0.11456130653340307,1.7522868926939459) node {$H$};
                    \draw [fill=black] (0,-1.5384615384615379) circle (0.6pt);
                    \draw[color=black] (0.11456130653340307-0.1,-1.3277987150640744-0.4) node {$D$};
                    \draw [fill=black] (0,4.019230769230769) circle (0.6pt);
                    \draw[color=black] (0.11456130653340307,4.232709815755714) node {$I$};
                    \draw [fill=black] (-11.030534351145043,6.946564885496185) circle (0.6pt);
                    \draw[color=black] (-7.1631630940983335-3.6,7.312795423513735) node {$M$};
                    \draw [fill=black] (-4.776859504132232,-3.008264462809916) circle (0.6pt);
                    \draw[color=black] (-4.178478367996554-0.4,-1.7911744259657236-1.3) node {$N$};
                    \draw [fill=black] (0.75,2.0096153846153846) circle (0.6pt);
                    \draw[color=black] (0.864139662403713+0.1,2.215662603595595-0.2) node {$J$};
                    \draw [fill=black] (0,0) circle (0.6pt);
                    \draw[color=black] (0.11456130653340307,0.21224408881493573-0.5) node {$K$};
                    \draw [fill=black] (3.7406559605306873,6.139140021554338) circle (0.6pt);
                    \draw[color=black] (3.8488243885054922,6.345157909572056) node {$P$};
                    \draw [fill=black] (-2.350844526286875,4.3715924405637825) circle (0.6pt);
                    \draw[color=black] (-2.243203340113208-0.2,4.587055947621681) node {$Q$};
                    \draw [fill=black] (-2.479974367190004,4.082024991989744) circle (0.6pt);
                    \draw[color=black] (-2.36586161652835-0.3,4.287224605273556-0.1) node {$R$};
                    \draw [fill=black] (1.6252989048528081,2.1450324216123473) circle (0.6pt);
                    \draw[color=black] (1.7363762946891645,2.3519495773901977) node {$S$};
                    \draw [fill=black] (0.5015874610670153,2.6752335979101773) circle (0.6pt);
                    \draw[color=black] (0.6051944121939695-0.1,2.883468775189148) node {$G$};
                    \draw [fill=black] (3,-1.5384615384615379) circle (0.6pt);
                    \draw[color=black] (3.1128747300146427,-1.3277987150640744) node {$L$};
                    \draw [fill=black] (1.5,0) circle (0.6pt);
                    \draw[color=black] (1.6137180182740227-0.1,0.21224408881493573-0.5) node {$T$};
                \end{scriptsize}
            \end{tikzpicture}
        \end{center}
        
        \begin{solution}
            \hfill
            \begin{itemize}
                \item[(a)] Gọi \((J)\) là đường tròn Euler của tam giác \(ABC\). Khi đó \(J\) là trung điểm của đoạn thẳng \(OH\).\\
                Do \(H\) và \(D\) đối xứng nhau qua \(BC\) nên \(\angle BDH = \angle BHD = \angle IHE = \angle IEH\). Suy ra tứ giác \(IBDE\) nội tiếp. Khi đó \(\overline{MB} \cdot \overline{MD} = \overline{MI} \cdot \overline{ME}\), hay \(\mathcal{P}_{M/(O)} = \mathcal{P}_{M/(J)}\).\\
                Tương tự, \(\mathcal{P}_{N/(O)} = \mathcal{P}_{N/(J)}\). Suy ra \(MN\) là trục đẳng phương của \((O)\) và \((J)\). Từ đó \(MN \perp OJ\), mà \(O\), \(J\), \(H\) thẳng hàng nên \(MN \perp OH\).

                \item[(b)] Gọi \(G\) là trung điểm của đoạn thẳng \(EF\); \(T\) là trung điểm của đoạn thẳng \(BC\); \(K\) là giao điểm của \(AD\) và \(BC\); \(L\) là điểm đối xứng với \(A\) qua \(O\).\\
                Không khó để chỉ ra rằng tứ giác \(BHCL\) là hình bình hành và \(R\), \(H\), \(T\), \(L\) thẳng hàng (các đường thẳng cùng vuông góc với \(AR\)). Ta chứng minh các đường thẳng \(BP\), \(CQ\) và \(RS\) đồng quy tại \(G\).
                \begin{itemize}
                    \item \textit{Chứng minh \(BP\) và \(CQ\) cùng đi qua điểm \(G\).}\\
                    Do tính đối xứng của bài toán, ta chỉ cần chỉ ra \(BP\) đi qua điểm \(G\); đối với việc chỉ ra \(CQ\) đi qua điểm \(G\) ta chứng minh tương tự.\\
                    Gọi \(G_1\) là giao điểm của \(BP\) và \(EF\). Xét các cặp tam giác \(BFG_1\) và \(DHE\), \(BFE\) và \(KHE\).\\
                    Ta có \(\angle BFG_1 = \angle ABP = \angle HDE\), \(\angle BEF = \angle BCF = \angle HEK\) và \(\angle BFE = 180 \degree - \angle AFE = 180 \degree - \angle AHE = \angle EHD\). Do đó \(\triangle BFG_1 \sim \triangle DHE\) và \(\triangle BFE \sim \triangle KHE\). Khi đó
                    \[\frac{\overline{BF}}{\overline{FE}} = \frac{\overline{KH}}{\overline{HE}} = \frac{\frac{\overline{DH}}{2}}{\overline{HE}} = \frac{\overline{BF}}{2\overline{FG_1}} \Rightarrow \overline{FE} = 2\overline{FG_1}\]
                    Suy ra \(G_1\) là trung điểm của đoạn thẳng \(EF\), từ đó \(G_1 \equiv G\). Vì vậy, \(BP\) đi qua điểm \(G\).
                    \item \textit{Chứng minh \(RS\) đi qua điểm \(G\).}\\
                    Gọi \(G_2\) là giao điểm của \(RS\) và \(EF\). Xét các cặp tam giác \(REF\) và \(CHL\), \(RFG_2\) và \(CLT\).\\
                    Ta có \(\angle RFE = \angle RHE = \angle BHL = \angle CLH\), \(\angle FRE = \angle FAE = \angle HCL\) và \(\angle FRG_2 = \angle FRS = \angle FAS = \angle TCL\). Do đó \(\triangle REF \sim \triangle CHL\) và \(\triangle RFG_2 \sim \triangle CLT\). Mà \(T\) là trung điểm của đoạn thẳng \(HL\) nên \(G_2\) là trung điểm của đoạn thẳng \(EF\), hay \(G_2 \equiv G\). Vì vậy, \(RS\) đi qua điểm \(G\).
                \end{itemize}
            \end{itemize}
            \hspace{0.8cm} Tóm lại, \(BP\), \(CQ\) và \(RS\) đồng quy.
        \end{solution}

        \begin{problem}
            Cho tam giác \(ABC\) có các đường cao \(AA'\), \(BB'\), \(CC'\). Gọi \(A_1\), \(B_1\), \(C_1\) là trung điểm các đoạn \(AA'\), \(BB'\), \(CC'\). Gọi \(B_a\), \(C_a\) là giao điểm của \(B_1C_1\) với \(AB\), \(AC\).
            \begin{enumerate}
                \item[(a)] Chứng minh rằng các đường tròn \((A'B_1C_1)\) và \((AB_aC_a)\) tiếp xúc nhau.
                \item[(b)] Gọi \(O_a\) là tâm đường tròn \((AB_aC_a)\). Chứng minh rằng \(A_1\), \(B_1\), \(C_1\), \(O_a\) đồng viên.
            \end{enumerate}
        \end{problem}

        \begin{center}
            \begin{tikzpicture}[line cap=round,line join=round,>=triangle 45,x=1cm,y=1cm,scale=0.7]
                \draw [line width=0.4pt] (0,7)-- (-2,0);
                \draw [line width=0.4pt] (-2,0)-- (5,0);
                \draw [line width=0.4pt] (5,0)-- (0,7);
                \draw [line width=0.4pt] (0,7)-- (0,0);
                \draw [line width=0.4pt] (-2,0)-- (2.635135135135135,3.310810810810811);
                \draw [line width=0.4pt] (5,0)-- (-1.471698113207547,1.849056603773585);
                \draw [line width=0.4pt] (-1.2943396226415096,2.4698113207547174)-- (5.793918918918919,-1.111486486486487);
                \draw [line width=0.4pt] (0,7)-- (-1.2943396226415096,2.4698113207547174);
                \draw [line width=0.4pt] (0,7)-- (5.793918918918919,-1.111486486486487);
                \draw [line width=0.4pt] (0.75,0.7142857142857143) circle (1.0357142857142858cm);
                \draw [line width=0.4pt] (3.675,3.5) circle (5.075cm);
                \draw [line width=0.4pt] (1.5,0)-- (0.31756756756756754,1.6554054054054055);
                \draw [line width=0.4pt] (1.5,0)-- (1.7641509433962264,0.9245283018867925);
                \draw [line width=0.4pt] (1.5,0)-- (0,1.4285714285714286);
                \draw [line width=0.4pt] (0,0)-- (-1.2943396226415096,2.4698113207547174);
                \draw [line width=0.4pt] (0,0)-- (5.793918918918919,-1.111486486486487);
                \draw [line width=0.4pt] (-1,1.05) circle (1.45cm);
                \draw [line width=0.4pt] (0,0)-- (0.31756756756756754,1.6554054054054055);
                \draw [line width=0.4pt] (0,0)-- (-5.965492855841051,6.2637674986331024);
                \draw [line width=0.4pt] (1.8375,2.866712454212454) circle (1.9435687190448432cm);
                \draw [line width=0.4pt] (-3.333333333333332,3.5)-- (3.675,3.5);
                \draw [line width=0.4pt] (-3.333333333333332,3.5)-- (-1.2943396226415096,2.4698113207547174);
                \draw [line width=0.4pt] (0,7)-- (-6.666666666666664,0);
                \draw [line width=0.4pt] (-6.666666666666664,0)-- (-2,0);
                \draw [line width=0.4pt] (-6.666666666666664,0)-- (2.635135135135135,3.310810810810811);
                \draw [line width=0.4pt] (1.5,2.7857142857142856) circle (4.473276660528907cm);
                \draw [line width=0.4pt] (0,1.4285714285714286)-- (-2.7824019024970297,4.078478002378118);
                \draw [line width=0.4pt] (0,4.214285714285715) circle (2.7857142857142847cm);
                \draw [line width=0.4pt] (1.5,0) circle (3.5cm);
                \begin{scriptsize}
                    \draw [fill=black] (0,7) circle (0.6pt);
                    \draw[color=black] (0.15700984452138245-0.1,7.289167199083975) node {$A$};
                    \draw [fill=black] (-2,0) circle (0.6pt);
                    \draw[color=black] (-1.8478081111916014-0.3,0.2912177310292787-0.5) node {$B$};
                    \draw [fill=black] (5,0) circle (0.6pt);
                    \draw[color=black] (5.150141356863154,0.2912177310292787-0.5) node {$C$};
                    \draw [fill=black] (0,0) circle (0.6pt);
                    \draw[color=black] (0.23266335228413654-0.25,0.2912177310292787-0.5) node {$A'$};
                    \draw [fill=black] (2.635135135135135,3.310810810810811) circle (0.6pt);
                    \draw[color=black] (2.8616227470398417-0.2,3.6010586956497437-0.6) node {$B'$};
                    \draw [fill=black] (-1.471698113207547,1.849056603773585) circle (0.6pt);
                    \draw[color=black] (-1.2425800490895684-0.3,2.144728671216739) node {$C'$};
                    \draw [fill=black] (0,3.5) circle (0.6pt);
                    \draw[color=black] (0.21374997534344803-0.4,3.8469325958786924-0.1) node {$A_1$};
                    \draw [fill=black] (0.31756756756756754,1.6554054054054055) circle (0.6pt);
                    \draw[color=black] (0.5163640063944644,2.0123350326319205-0.1) node {$B_1$};
                    \draw [fill=black] (1.7641509433962264,0.9245283018867925) circle (0.6pt);
                    \draw[color=black] (1.972694030827481,1.274713331945074-0.1) node {$C_1$};
                    \draw [fill=black] (-1.2943396226415096,2.4698113207547174) circle (0.6pt);
                    \draw[color=black] (-1.0912730335640604-0.3,2.8256102410815207) node {$B_a$};
                    \draw [fill=black] (5.793918918918919,-1.111486486486487) circle (0.6pt);
                    \draw[color=black] (6.001243319194137,-0.76793137764927-0.5) node {$C_a$};
                    \draw [fill=black] (0,1.4285714285714286) circle (0.6pt);
                    \draw[color=black] (0.15700984452138245,1.728634378521595) node {$H$};
                    \draw [fill=black] (1.5,0) circle (0.6pt);
                    \draw[color=black] (1.651166622835776,0.2912177310292787-0.5) node {$M$};
                    \draw[color=black] (-5.743963760973438,6.513718744515752) node {$x$};
                    \draw [fill=black] (3.675,3.5) circle (0.6pt);
                    \draw[color=black] (3.8829451018370222,3.8469325958786924-0.1) node {$O_a$};
                    \draw [fill=black] (-3.333333333333332,3.5) circle (0.6pt);
                    \draw[color=black] (-3.190657873980487-0.4,3.790192465056627-0.3) node {$T$};
                    \draw [fill=black] (-6.666666666666664,0) circle (0.6pt);
                    \draw[color=black] (-6.519412215541667-0.2,0.2912177310292787-0.5) node {$S$};
                    \draw [fill=black] (-2.7824019024970297,4.078478002378118) circle (0.6pt);
                    \draw[color=black] (-2.623256565759831-0.4,4.376507150217966-0.3) node {$R$};
                \end{scriptsize}
            \end{tikzpicture}
        \end{center}

        \begin{solution}
            \hfill
            \begin{enumerate}
                \item[(a)] Gọi \(H\) là trực tâm của tam giác \(ABC\); \(M\) là trung điểm của đoạn thẳng \(BC\). Tại \(A'\), kẻ \(A'x\) là tiếp tuyến của đường tròn \((A'B_aC_a)\).\\
                Sử dụng tính chất của đường trung bình trong các tam giác \(BB'C\) và \(CC'B\), ta được \(\angle HA'M = \angle HB_1M = \angle HC_1M = 90 \degree\). Do đó \(A'\), \(H\), \(B_1\), \(C_1\), \(M\) cùng thuộc đường tròn đường kính \(HM\). Khi đó, \(\angle A'B_1C_1 = \angle CMC_1 = \angle A'BC'\), suy ra \(A'\), \(B\), \(B_a\), \(B_1\) đồng viên.\\
                Do đó \(\angle A'B_aC = \angle A'B_aB_1 = \angle A'BB_1 = \angle CBB' = \angle A'AC_a\), suy ra \(A\), \(B_a\), \(A'\), \(C_a\) đồng viên. Từ đó
                \begin{equation}
                    \begin{aligned}
                        \angle HA'x &= \angle AA'x = \angle B_aA'x + \angle B_aA'A = \angle A'AB_a + \angle B_aA'A \\
                        &= 180 \degree - \angle AB_aA' = \angle BB_aA' = \angle HB_1A'.
                    \end{aligned}
                    \notag
                \end{equation}
                Suy ra \(A'x\) cũng là tiếp tuyến của đường tròn \((A'B_1C_1)\). Vì vậy, các đường tròn \((A'B_1C_1)\) và \((AB_aC_a)\) tiếp xúc nhau.
                \item[(b)] Gọi \(R\) là giao điểm của hai đường tròn \((ABC)\) và \((AH)\) (khác \(A\)); \(S\) là giao điểm của \(BC\) và tiếp tuyến tại \(A\) của đường tròn \((O_a)\); \(T\) là trung điểm của đoạn thẳng \(AS\).\\
                Khi đó \(TA_1 \perp AA'\). Lại có \(AA'\) là dây cung của \((O_a)\) nên \(OA_1 \perp AA'\). Suy ra \(T\), \(A_1\), \(O_a\) thẳng hàng và \(TA'\) là tiếp tuyến tại \(A'\) của \(O_a\).\\
                Ta có \(M\), \(H\), \(R\) thẳng hàng (cùng vuông góc \(AR\)) và \(A\), \(R\), \(A'\), \(M\) cùng thuộc đường tròn đường kính \(AM\). Khi đó \(\angle RAH = \angle A'MH = \angle AC_aA'\) hay \(AR\) là tiếp tuyến của \((O_a)\). Do đó \(A\), \(R\), \(T\), \(S\) thẳng hàng.\\
                Ngoài ra, \(M\) chính là tâm đường tròn ngoại tiếp tứ giác \(BC'B'C\). Theo tính chất tâm đẳng phương, trục đẳng phương của các cặp đường tròn \((ABC)\) và \((AH)\), \((ABC)\) và \((M)\), \((AH)\) và \((M)\) đồng quy. Nói cách khác, \(AR\), \(B'C'\), \(BC\) đồng quy tại \(S\), hay \(B'\), \(C'\), \(S\) thẳng hàng.\\
                Khi đó, theo tính chất của tứ giác toàn phần, \((S,A';B,C) = -1\). Chiếu xuyên tâm \(A\) lên đường tròn \((O_a)\), ta được tứ giác \(AB_aA'C_a\) là tứ giác điều hòa. Do đó \(T\), \(B_1\), \(C_1\) thẳng hàng. Khi đó \(\overline{TB_1} \cdot \overline{TC_1} = TA'^2 = TA^2 = \overline{TA_1} \cdot \overline{TO_a}\). Từ đó, \(A_1\), \(B_1\), \(C_1\), \(O_a\) đồng viên.
            \end{enumerate}
        \end{solution}

        \begin{problem}
            Cho tam giác nhọn không cân \(ABC\) có trực tâm \(H\) và \(D\), \(E\), \(F\) lần lượt là chân đường cao hạ từ các đỉnh \(A\), \(B\), \(C\). Gọi \(I\) là tâm đường tròn ngoại tiếp tam giác \(HEF\), và \(K\), \(J\) lần lượt là trung điểm các đoạn thẳng \(BC\), \(EF\). \(HJ\) cắt đường tròn \((I)\) tại \(G\) khác \(H\), \(GK\) cắt đường tròn \(I\) tại \(L\) khác \(G\).
            \begin{enumerate}
                \item[(a)] Chứng minh rằng \(AL\) vuông góc với \(EF\).
                \item[(b)] Cho \(AL\) cắt \(EF\) tại \(M\), \(IM\) cắt đường tròn ngoại tiếp tam giác \(IEF\) tại \(N\) khác \(I\). \(DN\) cắt \(AB\), \(AC\) tương ứng tại \(P\), \(Q\). Chứng minh rằng \(PE\), \(QF\), \(AK\) đồng quy.
            \end{enumerate}
        \end{problem}

        \begin{center}
            \begin{tikzpicture}[line cap=round,line join=round,>=triangle 45,x=1cm,y=1cm,scale=1.2]
                \draw [line width=0.4pt] (0,6.25)-- (-2,0);
                \draw [line width=0.4pt] (-2,0)-- (5,0);
                \draw [line width=0.4pt] (5,0)-- (0,6.25);
                \draw [line width=0.4pt] (2.268292682926829,3.4146341463414633)-- (-1.3497822931785195,2.0319303338171264);
                \draw [line width=0.4pt] (0,3.925) circle (2.325cm);
                \draw [line width=0.4pt] (0,6.25)-- (0,0);
                \draw [line width=0.4pt] (-2,0)-- (2.268292682926829,3.4146341463414633);
                \draw [line width=0.4pt] (5,0)-- (-1.3497822931785195,2.0319303338171264);
                \draw [line width=0.4pt] (0,1.6)-- (1.6288760134205988,5.584032227807228);
                \draw [line width=0.4pt] (1.6288760134205988,5.584032227807228)-- (1.5,0);
                \draw [line width=0.4pt] (0,6.25)-- (1.5506035611880067,2.1925873482247162);
                \draw [line width=0.4pt] (0,3.925)-- (1.5,0);
                \draw [line width=0.4pt] (1.5,0)-- (2.268292682926829,3.4146341463414633);
                \draw [line width=0.4pt] (1.5,0)-- (-1.3497822931785195,2.0319303338171264);
                \draw [line width=0.4pt] (1.5506035611880067,2.1925873482247162)-- (0,3.925);
                \draw [line width=0.4pt] (0,1.6)-- (1.5506035611880067,2.1925873482247162);
                \draw [line width=0.4pt] (0.75,1.9625) circle (2.1009298536600407cm);
                \draw [line width=0.4pt] (0,3.925)-- (2.847171197648787,1.836884643644379);
                \draw [line width=0.4pt] (-2,0)-- (-2.5203252032520327,-1.626016260162602);
                \draw [line width=0.4pt] (5,0)-- (3.2978723404255317,2.1276595744680855);
                \draw [line width=0.4pt] (-2.5203252032520327,-1.626016260162602)-- (3.2978723404255317,2.1276595744680855);
                \draw [line width=0.4pt] (0,6.25)-- (1.5,0);
                \draw [line width=0.4pt] (-2.5203252032520327,-1.626016260162602)-- (2.268292682926829,3.4146341463414633);
                \draw [line width=0.4pt] (3.2978723404255317,2.1276595744680855)-- (-1.3497822931785195,2.0319303338171264);
                \draw [line width=0.4pt] (2.268292682926829,3.4146341463414633)-- (9.6875,6.25);
                \draw [line width=0.4pt] (9.6875,6.25)-- (3.2978723404255317,2.1276595744680855);
                \draw [line width=0.4pt] (0,6.25)-- (9.6875,6.25);
                \draw [line width=0.4pt] (0,6.25)-- (2.847171197648787,1.836884643644379);
                \draw [line width=0.4pt] (0,0)-- (2.268292682926829,3.4146341463414633);
                \draw [line width=0.4pt] (0,0)-- (-1.3497822931785195,2.0319303338171264);
                \begin{scriptsize}
                    \draw [fill=black] (0,6.25) circle (0.6pt);
                    \draw[color=black] (0.14840150137787697-0.1,6.5130593329347555) node {$A$};
                    \draw [fill=black] (-2,0) circle (0.6pt);
                    \draw[color=black] (-1.8652320800541728-0.4,0.27955015911023495-0.5) node {$B$};
                    \draw [fill=black] (5,0) circle (0.6pt);
                    \draw[color=black] (5.138710811883392,0.27955015911023495-0.5) node {$C$};
                    \draw [fill=black] (0,0) circle (0.6pt);
                    \draw[color=black] (0.14840150137787697,0.27955015911023495-0.5) node {$D$};
                    \draw [fill=black] (2.268292682926829,3.4146341463414633) circle (0.6pt);
                    \draw[color=black] (2.4071730840277414,3.6939723189298457) node {$E$};
                    \draw [fill=black] (-1.3497822931785195,2.0319303338171264) circle (0.6pt);
                    \draw[color=black] (-1.2173673625499482-0.3,2.310693597772157-0.5) node {$F$};
                    \draw [fill=black] (0,1.6) circle (0.6pt);
                    \draw[color=black] (0.14840150137787697-0.3,1.8729471670260534) node {$H$};
                    \draw [fill=black] (0,3.925) circle (0.6pt);
                    \draw[color=black] (0.14840150137787697,4.201758178595327) node {$I$};
                    \draw [fill=black] (1.5,0) circle (0.6pt);
                    \draw[color=black] (1.6367393659146097,0.27955015911023495-0.5) node {$K$};
                    \draw [fill=black] (0.4592551948741548,2.723282240079295) circle (0.6pt);
                    \draw[color=black] (0.6036577893538186-0.15,2.9935780297360792) node {$J$};
                    \draw [fill=black] (1.6288760134205988,5.584032227807228) circle (0.6pt);
                    \draw[color=black] (1.7768182237533607,5.8476847582006775) node {$G$};
                    \draw [fill=black] (1.5506035611880067,2.1925873482247162) circle (0.6pt);
                    \draw[color=black] (1.6892689376041412,2.4682823128407545) node {$L$};
                    \draw [fill=black] (1.2345569754681582,3.0195759141916527) circle (0.6pt);
                    \draw[color=black] (1.3740915074669506,3.29124560264343) node {$M$};
                    \draw [fill=black] (2.847171197648787,1.836884643644379) circle (0.6pt);
                    \draw[color=black] (2.9849983726125906,2.1005753110140275) node {$N$};
                    \draw [fill=black] (-2.5203252032520327,-1.626016260162602) circle (0.6pt);
                    \draw[color=black] (-2.3730179397196465-0.4,-1.3488665632652719-0.2) node {$P$};
                    \draw [fill=black] (3.2978723404255317,2.1276595744680855) circle (0.6pt);
                    \draw[color=black] (3.4402546605885322,2.398242883921378) node {$Q$};
                    \draw [fill=black] (1.000717360114778+0.06,2.0803443328550935-0.22) circle (0.6pt);
                    \draw[color=black] (1.14646336347898+0.1,2.3457133122318456-0.6) node {$R$};
                    \draw [fill=black] (9.6875,6.25) circle (0.6pt);
                    \draw[color=black] (9.83135254948156,6.5130593329347555) node {$T$};
                    \draw [fill=black] (0.8138856476079346,2.858809801633605) circle (0.6pt);
                    \draw[color=black] (0.9538549339506969,3.1336568875748325) node {$S$};
                \end{scriptsize}
            \end{tikzpicture}
        \end{center}

        \begin{solution}
            \hfill
            \begin{enumerate}
                \item[(a)] Không khó để chỉ ra rằng \(I\), \(J\), \(K\) thẳng hàng và \(KE\), \(KF\) là tiếp tuyến tại \(E\), \(F\) của đường tròn \((I)\). Do \(GL\) đi qua giao của hai tiếp tuyến tại \(E\) và \(F\) của \((I)\) nên tứ giác \(GELF\) là tứ giác điều hòa, hay \(GL\) là đường đối trung của tam giác \(GEF\). Lại có \(J\) là trung điểm \(EF\) nên \(GL\), \(GJ\) là hai đường đẳng giác trong góc \(EGF\) (trung tuyến - đường đối trung). Tức là \(\angle EGL = \angle FGJ = \angle FGH\), suy ra \(HL \parallel EF\). Mà \(AL \perp HL\) nên \(AL \perp EF\).
                \item[(b)] Để thuận tiện, ta phát biểu và chứng minh bổ đề sau:
                \begin{lemma}
                    Cho tam giác \(ABC\). Gọi \(I\), \(I_a\) lần lượt là tâm đường tròn nội tiếp và tâm đường tròn bàng tiếp ứng với góc \(A\) trong tam giác \(ABC\). \(AI\) cắt \((ABC)\) tại \(D\) khác \(A\). \(H\) là tiếp điểm của \(BC\) và đường tròn bàng tiếp \((I_a)\). \(DH\) cắt \((ABC)\) tại \(P\) khác \(D\). Khi đó \(\angle API_a = 90 \degree\).
                \end{lemma}
                \begin{center}
                    \begin{tikzpicture}[line cap=round,line join=round,>=triangle 45,x=1cm,y=1cm,scale=0.4]
                        \draw [line width=0.4pt] (0,6)-- (-2,0);
                        \draw [line width=0.4pt] (-2,0)-- (5,0);
                        \draw [line width=0.4pt] (5,0)-- (0,6);
                        \draw [line width=0.4pt] (1.5,2.1666666666666665) circle (4.1163630117428225cm);
                        \draw [line width=0.4pt] (2.242847177784948,-5.886636007867582)-- (-2,0);
                        \draw [line width=0.4pt] (2.242847177784948,-5.886636007867582)-- (5,0);
                        \draw [line width=0.4pt] (2.242847177784948,-5.886636007867582)-- (2.242847177784948,0);
                        \draw [line width=0.4pt] (0,6)-- (1.5,-1.949696345076156);
                        \draw [line width=0.4pt] (1.5,-1.949696345076156)-- (4.239098853909303,5.239414933765222);
                        \draw [line width=0.4pt] (4.239098853909303,5.239414933765222)-- (0,6);
                        \draw [line width=0.4pt] (4.239098853909303,5.239414933765222)-- (2.242847177784948,-5.886636007867582);
                        \draw [line width=0.4pt] (1.5,-1.949696345076156)-- (2.242847177784948,-5.886636007867582);
                        \draw [line width=0.4pt] (0,6)-- (2.242847177784948,0);
                        \draw [line width=0.4pt] (1.5,-1.949696345076156)-- (-2,0);
                        \draw [line width=0.4pt] (1.5,-1.949696345076156)-- (5,0);
                        \draw [line width=0.4pt] (1.5,-1.949696345076156) circle (4.006409344787839cm);
                        \draw [line width=0.4pt] (0.7571528222150522,1.9872433177152686)-- (5,0);
                        \draw [line width=0.4pt] (0.7571528222150522,1.9872433177152686)-- (-2,0);
                        \begin{scriptsize}
                            \draw [fill=black] (0,6) circle (0.6pt);
                            \draw[color=black] (0.2039225708856303-0.1,6.384446318374606) node {$A$};
                            \draw [fill=black] (-2,0) circle (0.6pt);
                            \draw[color=black] (-1.8153011788028166-0.6,0.37485182525425687-0.5) node {$B$};
                            \draw [fill=black] (5,0) circle (0.6pt);
                            \draw[color=black] (5.203905189161784+0.2,0.37485182525425687-0.5) node {$C$};
                            \draw [fill=black] (0.7571528222150522,1.9872433177152686) circle (0.6pt);
                            \draw[color=black] (0.9491122880325571,2.370037196970213) node {$I$};
                            \draw [fill=black] (2.242847177784948,-5.886636007867582) circle (0.6pt);
                            \draw[color=black] (2.511606856243855,-5.442435644086241-0.8) node {$I_a$};
                            \draw [fill=black] (2.242847177784948,0) circle (0.6pt);
                            \draw[color=black] (2.4394917223264105+0.2,0.37485182525425687-0.7) node {$H$};
                            \draw [fill=black] (1.5,-1.949696345076156) circle (0.6pt);
                            \draw[color=black] (1.6943020051794837-0.5,-1.5722567905167364-0.7) node {$D$};
                            \draw [fill=black] (4.239098853909303,5.239414933765222) circle (0.6pt);
                            \draw[color=black] (4.434677094042375,5.6152182232552015) node {$P$};
                        \end{scriptsize}
                    \end{tikzpicture}
                \end{center}
                \begin{proof}
                    Không khó để chỉ ra rằng \(D\) là tâm đường tròn ngoại tiếp tam giác \(BIC\) và \(I_a\) thuộc đường tròn \(D\). Hơn nữa, theo tính chất của phân giác và xử lí cắc cặp tam giác đồng dạng, ta thu được \(DI_a^2 = DB^2 = DC^2 = \overline{DH} \cdot \overline{DP}\). Khi đó \(\angle DPI_a = \angle DI_aH\).\\
                    Mặt khác, \(I_aH\) và \(I_aD\) là hai đường đẳng giác trong góc \(BI_aC\) (đường cao - đường kính). Như vậy
                    \[\angle APD = \angle ACD = \angle ICB + \angle ICD = \angle II_aB + \angle I_aIC = \angle HI_aC + \angle DIC.\]
                    Từ đó \(\angle API_a = \angle APD + \angle DPI_a = \angle HI_aC + \angle DIC + \angle DI_aH = 90 \degree\).
                \end{proof}
                Trở lại bài toán.\\
                Gọi \(R\) và \(S\) là giao của \(AK\) với \((I)\) và \(EF\); \(T\) là giao của \(DN\) và tiếp tuyến tại \(A\) của \((I)\).\\
                Nhận thấy rằng \(H\) và \(A\) lần lượt là tâm đường tròn nội tiếp và tâm đường tròn bàng tiếp ứng với góc \(H\) trong tam giác \(HEF\). Hơn nữa, \(M\) là tiếp điểm của đường tròn bàng tiếp \((A)\) trên \(EF\) và \(N\) là giao của \(IM\) và \((HEF)\). Do đó, sử dụng bổ đề trên ta thu được \(\angle DNA = 90 \degree\). Lại có \(\angle DAT = 90 \degree\) nên \(TA^2 = \overline{TN} \cdot \overline{TD}\), hay \(\mathcal{P}_{T/(I)} = \mathcal{P}_{T/(IEF)}\). Nói cách khác, \(T\) nằm trên trục đẳng phương của hai đường tròn \((I)\) và \((IEF)\), hay \(T\), \(E\), \(F\) thẳng hàng.\\
                Do \(AR\) đi qua giao của hai tiếp tuyến tại \(E\) và \(F\) của \((I)\) nên tứ giác \(AERF\) là tứ giác điều hòa, hay \((E,F;A,R) = -1\). Chiếu xuyên tâm \(A\) lên đường thẳng \(TF\), ta thu được \((E,F;T,S) = -1\). Theo tính chất của tứ giác toàn phần, \(PE\), \(QF\), \(AK\) đồng quy.
            \end{enumerate}
        \end{solution}

        \begin{problem}
            Cho tứ giác \(ABCD\) không là hình thang nội tiếp đường tròn \(\omega\). Gọi \(P\), \(Q\), \(R\) lần lượt là giao điểm của \(AB\) và \(CD\), \(AD\) và \(BC\), \(AC\) và \(BD\). Gọi \(M\) là trung điểm của đoạn thẳng \(PQ\). \(MR\) cắt \(\omega\) tại \(K\). Chứng minh rằng đường tròn ngoại tiếp tam giác \(KPQ\) và \(\omega\) tiếp xúc nhau.
        \end{problem}

        \begin{center}
            \begin{tikzpicture}[line cap=round,line join=round,>=triangle 45,x=1cm,y=1cm,scale=0.9]
                \draw [line width=0.4pt] (0,0) circle (1.5cm);
                \draw [line width=0.4pt] (-6.119459214473732,-0.1951634271116116)-- (-0.46413362409799686,3.0243718890835987);
                \draw [line width=0.4pt] (-2.29919329863773,-0.32896845201912395) circle (3.8226084618214293cm);
                \draw [line width=0.4pt] (-6.119459214473732,-0.1951634271116116)-- (0.46449671540685633,1.4262688390959966);
                \draw [line width=0.4pt] (-6.119459214473732,-0.1951634271116116)-- (1.4691481879008754,-0.3026608696009015);
                \draw [line width=0.4pt] (-0.46413362409799686,3.0243718890835987)-- (1.4691481879008754,-0.3026608696009015);
                \draw [line width=0.4pt] (-0.46413362409799686,3.0243718890835987)-- (-1.4771317388121554,-0.26092494360213697);
                \draw [line width=0.4pt] (-3.2917964192858644,1.4146042309859936)-- (1.4848778882222777,0.2124562474217931);
                \draw [line width=0.4pt] (-1.0737015733578175,1.0474564102476758)-- (1.4691481879008754,-0.3026608696009015);
                \draw [line width=0.4pt] (0.46449671540685633,1.4262688390959966)-- (-1.4771317388121554,-0.26092494360213697);
                \draw [line width=0.4pt] (-1.4771317388121554,-0.26092494360213697)-- (1.4848778882222777,0.2124562474217931);
                \draw [line width=0.4pt] (-0.46413362409799686,3.0243718890835987)-- (0.9660541209747336,3.8385672666030266);
                \draw [line width=0.4pt] (0,0)-- (0.9660541209747336,3.8385672666030266);
                \draw [line width=0.4pt] (0,0)-- (-1.4139294515081986,2.483660101467701);
                \draw [line width=0.4pt] (0.9660541209747336,3.8385672666030266)-- (1.4848778882222777,0.2124562474217931);
                \draw [line width=0.4pt] (0.9660541209747336,3.8385672666030266)-- (-1.20741537838043,0.8900270243371506);
                \draw [line width=0.4pt] (-3.0970465912767,1.0725133458445049) circle (3.2774963715545438cm);
                \draw [line width=0.4pt] (0,0)-- (-6.119459214473732,-0.1951634271116116);
                \draw [line width=0.4pt] (0,0)-- (-0.46413362409799686,3.0243718890835987);
                \draw [line width=0.4pt] (-3.2917964192858644,1.4146042309859936)-- (-6.1940931825534,2.1450266916890097);
                \draw [line width=0.4pt] (0,0)-- (-6.1940931825534,2.1450266916890097);
                \draw [line width=0.4pt] (-6.1940931825534,2.1450266916890097)-- (-0.46413362409799686,3.0243718890835987);
                \draw [line width=0.4pt] (-6.1940931825534,2.1450266916890097)-- (-6.119459214473732,-0.1951634271116116);
                \begin{scriptsize}
                    \draw [fill=black] (0,0) circle (0.6pt);
                    \draw[color=black] (0.1524904325484101,0.28662948055288306-0.1) node {$O$};
                    \draw [fill=black] (-1.0737015733578175,1.0474564102476758) circle (0.6pt);
                    \draw[color=black] (-0.9269874486550629-0.3,1.3471691533141894) node {$A$};
                    \draw [fill=black] (0.46449671540685633,1.4262688390959966) circle (0.6pt);
                    \draw[color=black] (0.6070074351603988,1.725933322157513) node {$B$};
                    \draw [fill=black] (1.4691481879008754,-0.3026608696009015) circle (0.6pt);
                    \draw[color=black] (1.6296706910373733+0.1,-0.016381854521775846-0.5) node {$C$};
                    \draw [fill=black] (-1.4771317388121554,-0.26092494360213697) circle (0.6pt);
                    \draw[color=black] (-1.324689825940553-0.3,0.04043277080472271-0.5) node {$D$};
                    \draw [fill=black] (-6.119459214473732,-0.1951634271116116) circle (0.6pt);
                    \draw[color=black] (-5.964550894271271-0.4,0.09724739613122126-0.5) node {$P$};
                    \draw [fill=black] (-0.46413362409799686,3.0243718890835987) circle (0.6pt);
                    \draw[color=black] (-0.32096477850574473,3.3167428312994724) node {$Q$};
                    \draw [fill=black] (-0.3894996560183294,0.6841817702829774) circle (0.6pt);
                    \draw[color=black] (-0.24521194473707997-0.1,0.9684049844708656) node {$R$};
                    \draw [fill=black] (-3.2917964192858644,1.4146042309859936) circle (0.6pt);
                    \draw[color=black] (-3.142757836388508,1.706995113715347) node {$M$};
                    \draw [fill=black] (1.4848778882222777,0.2124562474217931) circle (0.6pt);
                    \draw[color=black] (1.6296706910373733,0.5138879818588773) node {$K$};
                    \draw [fill=black] (-1.20741537838043,0.8900270243371506) circle (0.6pt);
                    \draw[color=black] (-1.0595549077502262-0.4,1.1767252773346937-0.4) node {$L$};
                    \draw [fill=black] (0.9660541209747336,3.8385672666030266) circle (0.6pt);
                    \draw[color=black] (1.118339063098886,4.131085794312618) node {$T$};
                    \draw [fill=black] (-1.4139294515081986,2.483660101467701) circle (0.6pt);
                    \draw[color=black] (-1.2678752006140543-0.1,2.786472994918819-0.1) node {$E$};
                    \draw [fill=black] (0.1387312549209236,0.551241635879472) circle (0.6pt);
                    \draw[color=black] (0.2850578916435735+0.1,0.8358375253757024-0.1) node {$F$};
                    \draw [fill=black] (-6.1940931825534,2.1450266916890097) circle (0.6pt);
                    \draw[color=black] (-6.040303728039936-0.3,2.4455852429598277) node {$S$};
                \end{scriptsize}
            \end{tikzpicture}
        \end{center}

        \begin{solution}
            Gọi \(L\) là giao điểm của \(MR\) và đường tròn \(\omega\) khác \(K\); \(T\) là giao của hai tiếp tuyến tại \(K\) và \(L\) của đường tròn \(\omega\); \(O\) là tâm đường tròn \(\omega\); \(E\) là giao của \(OR\) và \(PQ\); \(F\) là giao của \(OT\) và \(KL\); \(S\) là điểm đối xứng của \(R\) qua \(M\).\\
            Áp dụng định lí Brocard cho tứ giác toàn phần \(ABCD.PQ\) có \(R\) là giao của \(AC\) và \(BD\) và \(O\) là tâm đường tròn ngoại tiếp tứ giác \(ABCD\), ta thu được \(R\) là trực tâm của tam giác \(OPQ\) và \(PQ\) là đối cực của \(R\). Nói cách khác, \(\overline{OR} \cdot \overline{OE} = R^2\), với \(R\) là bán kính của đường tròn \(\omega\). Lại có \(\overline{OF} \cdot \overline{OT} = R^2\) và \(\angle RFO = 90 \degree\) nên \(OR \perp ET\), hay \(T\) thuộc đường thẳng \(PQ\).\\
            Mặt khác, do \(S\) là điểm đối xứng của \(R\) (trực tâm tam giác \(OPQ\)) qua \(M\) nên \(OS\) là đường kính đường tròn ngoại tiếp tam giác \(OPQ\). Khi đó, do \(\angle OFS = \angle OPS = \angle OQS = 90 \degree\) nên \(F\) thuộc đường tròn \((OPQ)\). Từ đó, \(\overline{TP} \cdot \overline{TQ} = \overline{TF} \cdot \overline{TO} = TK^2\), hay \(TK\) cũng là tiếp tuyến của đường tròn \((KPQ)\). Vì vậy, đường tròn \((KPQ)\) và \(\omega\) tiếp xúc nhau.
        \end{solution}

        \begin{problem}
            Cho tam giác \(ABC\) nội tiếp đường tròn \((O)\), đường cao \(AH\) (\(H\) thuộc \(BC\)). Lấy điểm \(D\) trên \(AH\), đường tròn đường kính \(AD\) cắt \((O)\), \(AB\), \(AC\) tại \(K\), \(M\), \(N\) tương ứng. \(BN\) và \(CM\) cắt nhau tại \(P\). \(AP\) cắt \(BC\) tại \(Q\). Chứng minh rằng \(KQ\) đi qua một điểm cố định.
        \end{problem}

        \begin{solution}
            Gọi \(T\) là giao điểm của \(KQ\) và \((O)\); \(S\) là giao điểm của \(MN\) và \(BC\).\\
            Nhận thấy rằng \(AD\) đi qua tâm đường tròn ngoại tiếp tam giác \(AMN\), mà \(AH\), \(AO\) là hai đường đẳng giác trong góc \(A\) nên \(AO \perp MN\), hay \(MN\) đối song \(BC\) trong góc \(A\). Do đó tứ giác \(M\), \(N\), \(B\), \(C\) đồng viên. Theo tính chất tâm đẳng phương, trục đẳng phương của các cặp đường tròn \((O)\) và \((AD)\), \((O)\) và \((BMNC)\), \((AD)\) và \((BMNC)\) đồng quy. Nói cách khác, \(AK\), \(MN\), \(BC\) đồng quy tại \(S\)\\
            Theo tính chất của tứ giác toàn phần, \((S,Q;B,C) = -1\). Chiếu xuyên tâm \(K\) lên đường tròn \((O)\), ta thu được tứ giác \(ABTC\) là tứ giác điều hòa. Mà tam giác \(ABC\) cố định và điểm \(T\) xác định duy nhất nên \(T\) cố định. Vì vậy, \(KQ\) đi qua một điểm cố định.
        \end{solution}

        \begin{center}
            \begin{tikzpicture}[line cap=round,line join=round,>=triangle 45,x=1cm,y=1cm,scale=0.7]
                \draw [line width=0.4pt] (0,6)-- (-2,0);
                \draw [line width=0.4pt] (-2,0)-- (5,0);
                \draw [line width=0.4pt] (5,0)-- (0,6);
                \draw [line width=0.4pt] (1.5,2.1666666666666665) circle (4.1163630117428225cm);
                \draw [line width=0.4pt] (0,3.260918000716005) circle (2.739081999283995cm);
                \draw [line width=0.4pt] (-4.377005336197352,0)-- (2.694179015689176,2.766985181172989);
                \draw [line width=0.4pt] (-4.377005336197352,0)-- (-2,0);
                \draw [line width=0.4pt] (-2,0)-- (2.694179015689176,2.766985181172989);
                \draw [line width=0.4pt] (5,0)-- (-1.643449199570397,1.0696524012888085);
                \draw [line width=0.4pt] (-2.608275499793392,2.424575298243424)-- (1.0196078431372557,-1.9215686274509802);
                \draw [line width=0.4pt] (0,6)-- (0,0);
                \draw [line width=0.4pt] (0,6)-- (-4.377005336197352,0);
                \draw [line width=0.4pt] (1.5,-0.572415332617329) circle (3.5464995859319384cm);
                \draw [line width=0.4pt] (0,6)-- (-0.5843949085004947,0);
                \draw [line width=0.4pt] (-2.608275499793392,2.424575298243424)-- (-2,0);
                \draw [line width=0.4pt] (-2.608275499793392,2.424575298243424)-- (5,0);
                \begin{scriptsize}
                    \draw [fill=black] (0,6) circle (0.6pt);
                    \draw[color=black] (0.1896132562601796-0.2,6.378705932165826) node {$A$};
                    \draw [fill=black] (-2,0) circle (0.6pt);
                    \draw[color=black] (-1.807673090284105-0.45,0.3630696741217301-0.6) node {$B$};
                    \draw [fill=black] (5,0) circle (0.6pt);
                    \draw[color=black] (5.18282912262089+0.075,0.3630696741217301-0.6) node {$C$};
                    \draw [fill=black] (1.5,2.1666666666666665) circle (0.6pt);
                    \draw[color=black] (1.6875780161683929,2.5267965495447053) node {$O$};
                    \draw [fill=black] (0,0) circle (0.6pt);
                    \draw[color=black] (0.1896132562601796,0.3630696741217301-0.6) node {$H$};
                    \draw [fill=black] (0,0.5218360014320097) circle (0.6pt);
                    \draw[color=black] (0.1896132562601796,0.8861684791690427-0.5) node {$D$};
                    \draw [fill=black] (-2.608275499793392,2.424575298243424) circle (0.6pt);
                    \draw[color=black] (-2.4258807689763833-0.45,2.788345952068362-0.2) node {$K$};
                    \draw [fill=black] (-1.643449199570397,1.0696524012888085) circle (0.6pt);
                    \draw[color=black] (-1.4510148141154826-0.2,1.433044502627597+0.2) node {$M$};
                    \draw [fill=black] (2.694179015689176,2.766985181172989) circle (0.6pt);
                    \draw[color=black] (2.876438936730467,3.145004228236984) node {$N$};
                    \draw [fill=black] (-0.49817202376206965,0.8852529358238115) circle (0.6pt);
                    \draw[color=black] (-0.3097083303758915,1.242826755337665) node {$P$};
                    \draw [fill=black] (-0.5843949085004947,0) circle (0.6pt);
                    \draw[color=black] (-0.4048172040208574-0.35,0.3630696741217301-0.6) node {$Q$};
                    \draw [fill=black] (1.0196078431372557,-1.9215686274509802) circle (0.6pt);
                    \draw[color=black] (1.2120336479435632,-1.5628850171888304) node {$T$};
                    \draw [fill=black] (-4.377005336197352,0) circle (0.6pt);
                    \draw[color=black] (-4.185394931408253-0.3,0.3630696741217301-0.6) node {$S$};
                \end{scriptsize}
            \end{tikzpicture}
        \end{center}

        \begin{problem}
            Trong một tứ giác lồi \(ABCD\), các đường thẳng \(AB\) và \(CD\) cắt nhau tại \(E\) và các đường thẳng \(AD\) và \(BC\) cắt nhau tại \(F\). Gọi \(P\) là giao điểm của các đường chéo \(AC\) và \(BD\). Đường tròn \(\omega_1\) đi qua \(D\) và tiếp xúc với \(AC\) tại \(P\). Đường tròn \(\omega_2\) đi qua \(C\) và tiếp xúc với \(BD\) tại \(P\). Gọi \(Q\) là giao điểm khác \(P\) của hai đường tròn \(\omega_1\) và \(\omega_2\). Chứng minh rằng đường vuông góc hạ từ \(P\) xuống đường thẳng \(EF\) đi qua tâm đường tròn ngoại tiếp tam giác \(XQY\).
        \end{problem}

        \begin{center}
            \begin{tikzpicture}[line cap=round,line join=round,>=triangle 45,x=1cm,y=1cm,scale=0.9]
                \draw [line width=0.4pt] (-1.35,2.8)-- (3,0);
                \draw [line width=0.4pt] (1.4,3.75)-- (-2.25,0);
                \draw [line width=0.4pt] (-1.0283376522248455,0.8338909107993926) circle (1.4791324967984927cm);
                \draw [line width=0.4pt] (1.175574531523818,0.7117253145845703) circle (1.9583363381822438cm);
                \draw [line width=0.4pt] (-2.25,0)-- (0.005728835136855272,7.017823042647996);
                \draw [line width=0.4pt] (3,0)-- (0.005728835136855272,7.017823042647996);
                \draw [line width=0.4pt] (3,0)-- (-9.455263157894736,0);
                \draw [line width=0.4pt] (-9.455263157894736,0)-- (0.005728835136855272,7.017823042647996);
                \draw [line width=0.4pt] (0.2786144508689202,1.3949739219549535) circle (1.9941702663485799cm);
                \draw [line width=0.4pt] (0.2786144508689202,1.3949739219549535)-- (-2.5087251810177125,5.152691632829098);
                \draw [line width=0.4pt] (-7.10422525503997,1.7439152237889344)-- (2.2304756263259033,1.8035727507986632);
                \draw [line width=0.4pt] (-1.5353291105404985,2.2234205449851165)-- (1.9242435148977053,2.521304261958503);
                \draw [line width=0.4pt] (-1.5353291105404985,2.2234205449851165)-- (2.2304756263259033,1.8035727507986632);
                \draw [line width=0.4pt] (-1.6783097360420896,1.7785919323134998)-- (1.9242435148977053,2.521304261958503);
                \draw [line width=0.4pt] (-7.10422525503997,1.7439152237889344)-- (-1.5353291105404985,2.2234205449851165);
                \draw [line width=0.4pt] (-1.35,2.8)-- (1.4,3.75);
                \draw [line width=0.4pt] (0.005728835136855272,7.017823042647996)-- (-0.32596038013652817,0);
                \begin{scriptsize}
                    \draw [fill=black] (-1.35,2.8) circle (0.6pt);
                    \draw[color=black] (-1.2483064210685892-0.2,2.9959802226157883) node {$A$};
                    \draw [fill=black] (1.4,3.75) circle (0.6pt);
                    \draw[color=black] (1.5026752296427153,3.947741646795535) node {$B$};
                    \draw [fill=black] (3,0) circle (0.6pt);
                    \draw[color=black] (3.1063280402469355,0.20588508871899508-0.5) node {$C$};
                    \draw [fill=black] (-2.25,0) circle (0.6pt);
                    \draw[color=black] (-2.1479165343343714-0.15,0.20588508871899508-0.5) node {$D$};
                    \draw [fill=black] (0.005728835136855272,7.017823042647996) circle (0.6pt);
                    \draw[color=black] (0.10762766269432852,7.220236406646446) node {$F$};
                    \draw [fill=black] (-0.2277631207010273,2.0776406294167526) circle (0.6pt);
                    \draw[color=black] (-0.12705323641848418+0.4,2.278899697548855-0.15) node {$P$};
                    \draw [fill=black] (-9.455263157894736,0) circle (0.6pt);
                    \draw[color=black] (-9.344797440460628-0.15,0.20588508871899508-0.5) node {$E$};
                    \draw [fill=black] (-1.5353291105404985,2.2234205449851165) circle (0.6pt);
                    \draw[color=black] (-1.4308360092674437-0.2,2.4223158025622418) node {$X$};
                    \draw [fill=black] (1.9242435148977053,2.521304261958503) circle (0.6pt);
                    \draw[color=black] (2.0241883387822988,2.722185840317505) node {$Y$};
                    \draw [fill=black] (-0.3701305971839381,-0.49072088734364033) circle (0.6pt);
                    \draw[color=black] (-0.27046934143186974-0.15,-0.28955236496361303-0.5) node {$Q$};
                    \draw [fill=black] (0.2786144508689202,1.3949739219549535) circle (0.6pt);
                    \draw[color=black] (0.38142204499261+0.2,1.6009326556673915-0.4) node {$O$};
                    \draw [fill=black] (-7.10422525503997,1.7439152237889344) circle (0.6pt);
                    \draw[color=black] (-6.997988449332501-0.05,1.9399161766081234) node {$T$};
                    \draw [fill=black] (-1.6783097360420896,1.7785919323134998) circle (0.6pt);
                    \draw[color=black] (-1.5742521142808292-0.2,1.9790296597935926) node {$K$};
                    \draw [fill=black] (2.2304756263259033,1.8035727507986632) circle (0.6pt);
                    \draw[color=black] (2.3370962042660492,2.005105315250572) node {$L$};
                    \draw [fill=black] (-0.32596038013652817,0) circle (0.6pt);
                    \draw[color=black] (-0.21831803051791135+0.1,0.20588508871899508) node {$J$};
                    \draw [fill=black] (-0.215501815111021,2.3370632121847) circle (0.6pt);
                    \draw[color=black] (-0.11401540868999457+0.1,2.5396562521186494) node {$M$};
                    \draw [fill=black] (-0.24146329393946944,1.7877747338320444) circle (0.6pt);
                    \draw[color=black] (-0.14009106414697378+0.1,1.9920674875220823-0.4) node {$N$};
                \end{scriptsize}
            \end{tikzpicture}
        \end{center}

        \begin{solution}
            Gọi \(O\) là tâm đường tròn ngoại tiếp tam giác \(QXY\); \(K\) là giao điểm của \(AD\) và \(YP\), \(L\) là giao điểm của \(BC\) và \(XP\); \(T\) là giao điểm của \(XY\) và \(KL\); \(FP\) cắt \(XY\) tại \(M\), cắt \(KL\) tại \(N\), cắt \(CD\) tại \(J\).\\
            Theo tính chất của tứ giác toàn phần, tứ giác toàn phần \(KXYL.FT\) có \((T,M;X,Y) = (T,N;K,L) = -1\), và tứ giác toàn phần \(ABCD.EF\) có \((E,J;D,C) = -1\). Mà \(AD\), \(PJ\), \(BC\) đồng quy tại \(F\) nên \(E\), \(T\), \(F\) thẳng hàng.\\
            Mặt khác, do \(\omega_1\) tiếp xúc \(AC\) tại \(P\) và \(\omega_2\) tiếp xúc \(BD\) tại \(P\) nên \(\angle LXK = \angle DPC = \angle KYL\). Do đó, \(X\), \(Y\), \(K\), \(L\) cùng thuộc đường tròn \((O)\). Áp dụng định lí Brocard cho tứ giác toàn phần \(KXYL.FT\) có \(P\) là giao của \(XL\) và \(YK\) và \(O\) là tâm ngoại tiếp tứ giác \(KXYL\), ta thu được \(OP \perp EF\).
        \end{solution}

        \begin{problem}
            Cho tam giác \(ABC\), các đường cao \(BE\) và \(CF\) cắt nhau tại trực tâm \(H\). \(K\) là điểm thay đổi trên đoạn thẳng \(AH\). Gọi \(M\), \(N\) là hình chiếu của \(H\) lên \(KE\), \(KF\). Chứng minh rằng đường thẳng đi qua tâm đường tròn ngoại tiếp các tam giác \(HEN\) và \(HFM\) luôn đi qua một điểm cố định.
        \end{problem}

        \begin{center}
            \begin{tikzpicture}[line cap=round,line join=round,>=triangle 45,x=1cm,y=1cm,scale=0.95]
                \draw [line width=0.4pt] (0,6.5)-- (-2,0);
                \draw [line width=0.4pt] (-2,0)-- (5,0);
                \draw [line width=0.4pt] (5,0)-- (0,6.5);
                \draw [line width=0.4pt] (0.5409318411178673,3.3160193757775422) circle (1.858041743809518cm);
                \draw [line width=0.4pt] (-0.408386337167144,2.691975173437551) circle (1.223671976658929cm);
                \draw [line width=0.4pt] (-2,0)-- (2.3977695167286246,3.382899628252788);
                \draw [line width=0.4pt] (5,0)-- (-1.3945945945945946,1.9675675675675677);
                \draw [line width=0.4pt] (0,3.727951013187709)-- (2.3977695167286246,3.382899628252788);
                \draw [line width=0.4pt] (0,3.727951013187709)-- (-1.3945945945945946,1.9675675675675677);
                \draw [line width=0.4pt] (0,1.5384615384615385)-- (0.30868632381599975,3.683529461074352);
                \draw [line width=0.4pt] (0,1.5384615384615385)-- (-1.0657036981983408,2.3827234068883425);
                \draw [line width=0.4pt] (0,6.5)-- (0,1.5384615384615385);
                \draw [line width=0.4pt] (0,3.727951013187709)-- (1.081863682235735,5.093577213093547);
                \draw [line width=0.4pt] (-6.666666666666676,0)-- (2.3977695167286246,3.382899628252788);
                \draw [line width=0.4pt] (0,1.5384615384615385)-- (-6.666666666666676,0);
                \draw [line width=0.4pt] (-6.666666666666676,0)-- (-2,0);
                \draw [line width=0.4pt] (1.081863682235735,5.093577213093547)-- (-6.666666666666676,0);
                \draw [line width=0.4pt] (0,1.5384615384615385)-- (1.081863682235735,5.093577213093547);
                \draw [line width=0.4pt] (0,1.5384615384615385)-- (-0.8167726743342886,3.845488808413566);
                \draw [line width=0.4pt] (0.5409318411178673,3.3160193757775422)-- (-3.3333333333333326,0.7692307692307694);
                \draw [line width=0.4pt] (0,3.727951013187709)-- (-0.8167726743342886,3.845488808413566);
                \draw [line width=0.4pt] (-3.3333333333333326,0.7692307692307694)-- (2.5,0.7692307692307693);
                \draw [line width=0.4pt] (0,1.5384615384615385)-- (0,0);
                \begin{scriptsize}
                    \draw [fill=black] (0,6.5) circle (0.6pt);
                    \draw[color=black] (0.11503032383230716-0.1,6.728557108445274) node {$A$};
                    \draw [fill=black] (-2,0) circle (0.6pt);
                    \draw[color=black] (-1.8790453444326212-0.25,0.21553298332102042-0.5) node {$B$};
                    \draw [fill=black] (5,0) circle (0.6pt);
                    \draw[color=black] (5.1217382966701495+0.05,0.21553298332102042-0.5) node {$C$};
                    \draw [fill=black] (2.3977695167286246,3.382899628252788) circle (0.6pt);
                    \draw[color=black] (2.510790299373624,3.6011578589362716) node {$E$};
                    \draw [fill=black] (-1.3945945945945946,1.9675675675675677) circle (0.6pt);
                    \draw[color=black] (-1.2765188835180385-0.4,2.195262783468922-0.075) node {$F$};
                    \draw [fill=black] (0,1.5384615384615385) circle (0.6pt);
                    \draw[color=black] (0.11503032383230716+0.05,1.7648867399585084-0.5) node {$H$};
                    \draw [fill=black] (0,3.727951013187709) circle (0.6pt);
                    \draw[color=black] (0.11503032383230716,3.9454586937446026+0.2) node {$K$};
                    \draw [fill=black] (0.30868632381599975,3.683529461074352) circle (0.6pt);
                    \draw[color=black] (0.41629355428959847,3.902421089393561) node {$M$};
                    \draw [fill=black] (-1.0657036981983408,2.3827234068883425) circle (0.6pt);
                    \draw[color=black] (-0.9465639168267195-0.1,2.6112929588623213) node {$N$};
                    \draw [fill=black] (0.5409318411178673,3.3160193757775422) circle (0.6pt);
                    \draw[color=black] (0.7032109166298759+0.15,3.5868119908192577-0.2) node {$O_1$};
                    \draw [fill=black] (-0.408386337167144,2.691975173437551) circle (0.6pt);
                    \draw[color=black] (-0.24361637909303968-0.05,2.955593793670652) node {$O_2$};
                    \draw [fill=black] (-1,0.7692307692307693) circle (0.6pt);
                    \draw[color=black] (-0.889180444358664-0.1,0.9902098616397643) node {$U$};
                    \draw [fill=black] (2.5,0.7692307692307693) circle (0.6pt);
                    \draw[color=black] (2.6112113761927214,0.9902098616397643) node {$V$};
                    \draw [fill=black] (1.081863682235735,5.093577213093547) circle (0.6pt);
                    \draw[color=black] (1.1909704326083477,5.322662032977925) node {$P$};
                    \draw [fill=black] (-3.3333333333333326,0.7692307692307694) circle (0.6pt);
                    \draw[color=black] (-3.2132110793149113,0.9902098616397643) node {$T$};
                    \draw [fill=black] (-0.8167726743342886,3.845488808413566) circle (0.6pt);
                    \draw[color=black] (-0.7026841588374836-0.175,4.074571506797726) node {$Q$};
                    \draw [fill=black] (-6.666666666666676,0) circle (0.6pt);
                    \draw[color=black] (-6.555798350579144-0.1,0.21553298332102042-0.5) node {$S$};
                    \draw [fill=black] (0,0) circle (0.6pt);
                    \draw[color=black] (0.11503032383230716-0.1,0.21553298332102042-0.5) node {$D$};
                    \draw [fill=black] (0,4.378793656731691) circle (0.6pt);
                    \draw[color=black] (0.11503032383230716,4.605368627127236+0.1) node {$R$};
                    \draw [fill=black] (0,2.4880382775119623) circle (0.6pt);
                    \draw[color=black] (0.11503032383230716,2.7117140356814176) node {$L$};
                \end{scriptsize}
            \end{tikzpicture}
        \end{center}

        \begin{solution}
            Gọi \(O_1\), \(O_2\) lần lượt là tâm đường tròn ngoại tiếp các tam giác \(HEN\) và \(HFM\); \(P\) là giao điểm của \(FK\) và \(AC\), \(Q\) là giao điểm của \(EK\) và \(AB\). Đường thẳng \(AH\) cắt \(BC\) tại \(D\), cắt \(EF\) tại \(L\), cắt \(PQ\) tại \(R\). Hơn nữa, gọi \(S_1\) là giao điểm của \(EF\) và \(BC\), \(S_2\) là giao điểm của \(EF\) và \(PQ\).\\
            Theo tính chất của tứ giác toàn phần, tứ giác toàn phần \(BFEC.AS_1\) có \((S_1,D;B,C) = (S_1,L;F,E) = -1\), và tứ giác toàn phần \(FQPE.AS_2\) có \((S_2,R;Q,P) = (S_2,L;F,E) = -1\). Khi đó \((S_1,L;F,E) = (S_2,L;F,E) = -1\), suy ra \(S_1 \equiv S_2 \equiv S\), hay \(BC\), \(EF\), \(PQ\) đồng quy tại \(S\).\\
            Mặt khác, do \(\angle HEP = \angle HNP = 90 \degree\) và \(\angle HFQ = \angle HMQ = 90 \degree\) nên \(HP\) và \(HQ\) lần lượt là đường kính của đường tròn \((O_1)\) và \((O_2)\).\\
            Gọi \(T\), \(U\), \(V\) tương ứng là trung điểm của các đoạn thẳng \(HS\), \(HB\), \(HC\). Xét phép vị tự tâm \(H\), tỉ số \(\dfrac{1}{2}\)
            \[\mathcal{H}_{H}^{\frac{1}{2}}: B \mapsto U, C \mapsto V, S \mapsto T, P \mapsto O_1, Q \mapsto O_2, TV \mapsto SC.\]
            Do đó phép vị tự trên biến \(PS\) thành \(O_1T\), hay \(O_1\), \(O_2\), \(T\) thẳng hàng. Lại có \(S\) là giao điểm của \(EF\) và \(BC\) nên \(S\) là điểm cố định. Mà \(H\) cố định và \(T\) là trung điểm đoạn thẳng \(SH\) nên \(T\) là điểm cố định.\\
            Tóm lại, tâm đường tròn ngoại tiếp các tam giác \(HEN\) và \(HFM\) luôn đi qua một điểm cố định.
        \end{solution}

        \begin{problem}
            Cho hình bình hành \(ABCD\) có \(\angle BAD > 90 \degree\). Gọi \(G\), \(E\), \(F\) là các điểm nằm trên các đường thẳng tương ứng là \(BD\), \(BC\), \(CD\) sao cho \(AG \perp AC\), \(AE \perp BC\), \(AF \perp CD\).
            \begin{enumerate}
                \item[(a)] Chứng minh rằng \(G\), \(E\), \(F\) thẳng hàng.
                \item[(b)] Đường cao ứng với đỉnh \(A\) của tam giác \(ABD\) cắt \(BC\) tại \(M\). Trung tuyến ứng với đỉnh \(A\) của tam giác \(AEF\) cắt \(CD\) tại \(N\). Chứng minh rằng \(EN\), \(FM\), và đường cao ứng với đỉnh \(A\) của tam giác \(AEF\) đồng quy.
            \end{enumerate}
        \end{problem}

        \begin{center}
            \begin{tikzpicture}[line cap=round,line join=round,>=triangle 45,x=1cm,y=1cm]
                \draw [line width=0.4pt] (2,3)-- (7,3);
                \draw [line width=0.4pt] (7,3)-- (5,0);
                \draw [line width=0.4pt] (5,0)-- (0,0);
                \draw [line width=0.4pt] (0,0)-- (2,3);
                \draw [line width=0.4pt] (5.461538461538462,0.6923076923076923)-- (2,3);
                \draw [line width=0.4pt] (2,3)-- (2,0);
                \draw [line width=0.4pt] (7,3)-- (0,0);
                \draw [line width=0.4pt] (2,3)-- (5,0);
                \draw [line width=0.4pt] (0,0)-- (-1.75,-0.75);
                \draw [line width=0.4pt] (2,3)-- (-1.75,-0.75);
                \draw [line width=0.4pt,dash pattern=on 3pt off 3pt] (-1.75,-0.75)-- (5.461538461538462,0.6923076923076923);
                \draw [line width=0.4pt] (2,3)-- (3.9565217391304346,-1.565217391304348);
                \draw [line width=0.4pt] (5,0)-- (3.9565217391304346,-1.565217391304348);
                \draw [line width=0.4pt] (2,3)-- (3.9565217391304355,0);
                \draw [line width=0.4pt,dash pattern=on 3pt off 3pt] (5.461538461538462,0.6923076923076923)-- (2.714285714285715,-0.5714285714285718);
                \draw [line width=0.4pt] (2,0)-- (3.9565217391304346,-1.565217391304348);
                \draw [line width=0.4pt] (2,3)-- (2.714285714285715,-0.5714285714285718);
                \draw [line width=0.4pt] (3.5,1.5) circle (2.121320343559643cm);
                \draw [line width=0.4pt] (-1.75,-0.75)-- (3.5517241379310347,-0.6206896551724139);
                \begin{scriptsize}
                    \draw [fill=black] (2,3) circle (0.6pt);
                    \draw[color=black] (2.092496162552361-0.2,3.1786868041218668) node {$A$};
                    \draw [fill=black] (7,3) circle (0.6pt);
                    \draw[color=black] (7.094208963166001,3.1786868041218668) node {$B$};
                    \draw [fill=black] (5,0) circle (0.6pt);
                    \draw[color=black] (5.089017795352425+0.05,0.17090005240148098-0.35) node {$C$};
                    \draw [fill=black] (0,0) circle (0.6pt);
                    \draw[color=black] (0.08730499473878468-0.075,0.17090005240148098-0.35) node {$D$};
                    \draw [fill=black] (3.5,1.5) circle (0.6pt);
                    \draw[color=black] (3.590756978952393,1.6691608688015234) node {$O$};
                    \draw [fill=black] (-1.75,-0.75) circle (0.6pt);
                    \draw[color=black] (-1.658788437907869-0.15,-0.5725977963383897) node {$G$};
                    \draw [fill=black] (5.461538461538462,0.6923076923076923) circle (0.6pt);
                    \draw[color=black] (5.550887671084765+0.125,0.8693374254601474-0.2) node {$E$};
                    \draw [fill=black] (2,0) circle (0.6pt);
                    \draw[color=black] (2.092496162552361,0.17090005240148098) node {$F$};
                    \draw [fill=black] (3.730769230769231,0.34615384615384615) circle (0.6pt);
                    \draw[color=black] (3.8160593573584127,0.5201187389308142) node {$J$};
                    \draw [fill=black] (3.9565217391304346,-1.565217391304348) circle (0.6pt);
                    \draw[color=black] (4.041361735764433+0.1,-1.3949514775203677-0.3) node {$M$};
                    \draw [fill=black] (3.9565217391304355,0) circle (0.6pt);
                    \draw[color=black] (4.041361735764433,0.17090005240148098) node {$N$};
                    \draw [fill=black] (2.714285714285715,-0.5714285714285718) circle (0.6pt);
                    \draw[color=black] (2.8021986545313236-0.1,-0.3923558936135726-0.45) node {$I$};
                    \draw [fill=black] (3.5517241379310347,-0.6206896551724139) circle (0.6pt);
                    \draw[color=black] (3.647082573553898,-0.4486814882150779) node {$L$};
                    \draw [fill=black] (2.7758620689655173,1.1896551724137931) circle (0.6pt);
                    \draw[color=black] (2.8697893680531297,1.3650026579533945) node {$H$};
                    \draw [fill=black] (2.576923076923077,0.11538461538461538) circle (0.6pt);
                    \draw[color=black] (2.6670172274877118,0.29481636052479276) node {$K$};
                \end{scriptsize}
            \end{tikzpicture}
        \end{center}

        \begin{solution}
            \hfill
            \begin{enumerate}
                \item[(a)] Gọi \(O\) là giao điểm của \(AC\) và \(BD\).\\
                Do \(AG \perp AC\), \(AE \perp BC \parallel AD\) và \(AF \perp CD \parallel AB\) nên ta có \(A(E,F;C,G) = A(D,B;G,C)\). Chiếu xuyên tâm \(A\) lên đường thẳng \(BD\), rồi chiếu xuyên tâm \(C\), ta được \(A(D,B;G,C) = (D,B;G,O) = C(D,B;G,O) = C(F,E;G,A) = C(E,F;A,G)\). Do đó \(G\), \(E\), \(F\) thẳng hàng.
                \item[(b)] Đường cao ứng với đỉnh \(A\) của tam giác \(AEF\) cắt \(FM\) tại \(I\) và \(EF\) tại \(K\). Đường cao ứng với đỉnh \(A\) của tam giác \(ABD\) cắt \(BD\) tại \(H\) và \((AEF)\) tại \(L\). Gọi \(J\) là trung điểm đoạn thẳng \(EF\).\\
                Do \(\angle AFC = \angle AEC = 90 \degree\) nên \(A\), \(F\), \(C\), \(E\) cùng thuộc đường tròn \((O)\). Do \(OH \perp AL\) và \(GA\) là tiếp tuyến tại \(A\) của \((O)\) nên \(GL\) là tiếp tuyến tại \(L\) của \((O)\). Mà \(G\), \(E\), \(F\) thẳng hàng nên tứ giác \(AELF\) điều hòa.\\
                Khi đó, \(AJ\) và \(AL\) là hai đường đẳng giác trong góc \(EAF\) (trung tuyến - đường đối trung). Ta cũng có \(AK\), \(AC\) là hai đường đẳng giác trong góc \(EAF\) (đường cao - đường kính). Xét phép đối xứng qua phân giác góc \(EAF\), biến \(AE\) thành \(AF\) và ngược lại, biến \(AJ\) thành \(AL\) và ngược lại, biến \(AK\) thành \(AC\) và ngược lại. Phép đối xứng này bảo toàn chùm tỉ số kép xuyên tâm \(A\), do đó
                \[A(F,C;E,M) = A(E,K;F,J) = A(E,I;F,N) = A(F,N;E,I).\]
                Mà \(F(A,N;E,I) = F(A,C;E,M) = A(F,C;E,M)\) nên \(F(A,N;E,I) = A(F,N;E,I)\). Từ đó \(E\), \(N\), \(I\) thẳng hàng, hay \(EN\), \(FM\), và đường cao ứng với đỉnh \(A\) của tam giác \(AEF\) đồng quy.
            \end{enumerate}
            Chứng minh hoàn tất.
        \end{solution}